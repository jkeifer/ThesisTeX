% *************** Document style definitions ***************

% ******************************************************************
% This file defines the document design.
% Usually it is not necessary to edit this file, but you can change
% the design if you want.
% ******************************************************************

% The origin of this file is not clear.  Here is one copy:
% https://subversion.cs.uu.nl/repos/staff.doaitse.wxFlashkell/MasterThesis/style_old.tex
% Emerson Murphy-Hill (emerson@cs.pdx.edu) has modified it.
% Jarrett Keifer (jkeifer@pdx.edu) has also modified it. It is now using AAG format for the references, suitable for all geography theses.


%************************* NOTES ON FORMATTING *************************

%Spacing:
%  Cannot use the setspace package with memoir class. However, memoir has its own spacing commands, and they are exactly the same, except use camel-casing:
%    \begin{Spacing}{1.5} or \DoubleSpacing or \OnehalfSpacing or \SingleSpacing


% *************** Load packages ***************
% *************** Colors ***************
\usepackage[usenames, dvipsnames]{color}
%\definecolor{bg}{rgb}{0.95,0.95,0.95} 
\definecolor{light-gray}{gray}{0.85}

% **************** Syntax Highlighting ******************
\usepackage{minted} %for code snippets; use \begin{minted}[mathescape, linenos, etc]{python}
\usemintedstyle{tango}


% ****************** For Figures and Graphics *******************
%\usepackage{fixltx2e}
%\usepackage{float}
\usepackage{graphicx}
\usepackage[justification=centering, labelfont=bf, small, compatibility=true]{caption} %caption package with centering, and can be 10pts, per thesis guidelines
\usepackage{subcaption} %allows subfigure captions
\captionsetup[sub]{font=footnotesize} %The subcaption settings
\captionsetup{compatibility=false} %To make caption package work
%\usepackage{cleveref} %Use \cref to add prefix to refs (i.e. Fig., Table, etc.)
%\usepackage{flafter} %Don't place floats until after refs


%************** NOTES *************
\usepackage{endnotes, chngcntr}
\counterwithin*{endnote}{chapter}  % Reset endnote numbering every new chapter

%To get chapters printed in endnotes
\makeatletter
\renewcommand\enoteheading{%
  \setcounter{secnumdepth}{-2}
  \chapter*{\notesname}
  \mbox{}\par\vskip-\baselineskip
  \let\@afterindentfalse\@afterindenttrue
}
\makeatother

\usepackage{xparse}

\let\latexchapter\chapter

\RenewDocumentCommand{\chapter}{som}{%
  \IfBooleanTF{#1}
    {\latexchapter*{#3}}
    {\IfNoValueTF{#2}
       {\latexchapter{#3}}
       {\latexchapter[#2]{#3}}%
     \addtoendnotes{%
       \noexpand\enotedivision{\noexpand\subsection*}
         {\chaptername\ \thechapter. \unexpanded{#3}}}%
    }%
}

\makeatletter
\def\enotedivision#1#2{\@ifnextchar\enotedivision{}{#1{#2}}}
\makeatletter

% Change \footnote to behave as \endnote
\let\footnote=\endnote

% Size of note to be same as rest of text
\renewcommand{\enotesize}{\normalsize}

% ***************** OTHERS *****************
\usepackage{varioref} 
\usepackage{times}
\usepackage{comment}
\usepackage{amssymb}
\usepackage{amsmath}
\usepackage{ifthen}
\usepackage{multirow}
\usepackage{xspace}
\usepackage{url}
\urlstyle{sf}
\usepackage{makeidx}
\usepackage{needspace}
\usepackage{tabularx}
\usepackage{colortbl}
\usepackage{enumitem}
\usepackage{afterpage}
\usepackage{longtable}
\usepackage{lscape}
\usepackage{ulem}
\usepackage{epsfig}
\usepackage{amsthm}
\usepackage{booktabs}
\usepackage{stmaryrd}
\usepackage[figuresright]{rotating}
\usepackage{xltxtra} %Xeletex Package
\usepackage{fontspec} %Font package
\usepackage{xunicode} %Xeletex

\normalem %normal emphasis for package ulem
\vrefwarning %warnings, not errors, for vrefs for varioref package


% *************** TABLES *****************
\usepackage{tabu}
\usepackage{array}
\newcolumntype{Z}{>{\centering\let\newline\\\arraybackslash\hspace{0pt}}X}

%To define \captionabove command for tables
\makeatletter
\newcommand{\captionabove}[2][]{%
    \vskip-\abovecaptionskip
    \vskip+\belowcaptionskip
    \ifx\@nnil#1\@nnil
        \caption{#2}%
    \else
        \caption[#1]{#2}%
    \fi
    \vskip+\abovecaptionskip
    \vskip-\belowcaptionskip
}

\usepackage{mathspec}  % loads fontspec as well
\usepackage{xltxtra}  % xelatex package
\usepackage{xunicode}  % xelatex unicode support?

%TODO MOVE ALL GROUPS OF SETTINGS (I.E. TABLES & FIGURES, DOCUMENT LAYOUT, ETC. TO NEW TEX FILES INSIDE A SETTINGS FOLDER]


% *************** Set Fonts ***************
\defaultfontfeatures{Ligatures=TeX}
%\setmainfont[BoldFont="Baskerville Semibold"]{Baskerville}
\setmainfont{Baskerville}
\setromanfont[Mapping=tex-text]{Baskerville}
\setsansfont[Mapping=tex-text]{Myriad Pro}
\newfontfamily\titleFont[Ligatures=TeX]{Myriad Pro}
\setmonofont{DejaVu Sans Mono}

\setmathrm{Baskerville}  % math roman font
\setmathfont(Digits,Latin){Baskerville}  % math digits and latin characters

% fix url digit issue
\makeatletter
     \DeclareMathSymbol{0}{\mathalpha}{\eu@DigitsArabic@symfont}{`0}
     \DeclareMathSymbol{1}{\mathalpha}{\eu@DigitsArabic@symfont}{`1}
     \DeclareMathSymbol{2}{\mathalpha}{\eu@DigitsArabic@symfont}{`2}
     \DeclareMathSymbol{3}{\mathalpha}{\eu@DigitsArabic@symfont}{`3}
     \DeclareMathSymbol{4}{\mathalpha}{\eu@DigitsArabic@symfont}{`4}
     \DeclareMathSymbol{5}{\mathalpha}{\eu@DigitsArabic@symfont}{`5}
     \DeclareMathSymbol{6}{\mathalpha}{\eu@DigitsArabic@symfont}{`6}
     \DeclareMathSymbol{7}{\mathalpha}{\eu@DigitsArabic@symfont}{`7}
     \DeclareMathSymbol{8}{\mathalpha}{\eu@DigitsArabic@symfont}{`8}
     \DeclareMathSymbol{9}{\mathalpha}{\eu@DigitsArabic@symfont}{`9}
\makeatother

\usepackage[final]{microtype}


% *************** Enable index generation ***************
\makeindex


% **************** Language/Localization ***************
\usepackage[australian, spanish, american]{babel} %datetime incompatible with polyglossia
%\usepackage{polyglossia}
%\setdefaultlanguage[variant=american]{english}
%\setotherlanguage{spanish}


% *************** Set Date Format ***************
\usepackage[nodayofweek]{datetime}
\setdefaultdate{\dateaustralian}  % formats dates of dd month yyyy without ordinal or comma

\newdateformat{noyear}{\THEDAY\ \monthname[\THEMONTH]}
\newdateformat{aagdate}{\THEDAY\ \monthname[\THEMONTH]\ \THEYEAR}

% command to print noyear dates, args {dayno}{monthno}
\newcommand{\datenoyear}[2]{%
  \noyear\formatdate{#1}{#2}{1}%
  \aagdate}


% ********* Reference Settings **********
% *************** Set Biblatex Style and Options ***************

\usepackage{csquotes}
\usepackage[authordate-trad, backend=biber, firstinits=true, maxcitenames=3, isbn=false, doi=false, eprint=false, shorthandfull, sorting=nyt, sortcites=true, ibidtracker=false, abbreviate=true]{biblatex-chicago}
\addbibresource[datatype=bibtex]{thesis.bib}

%Check for page range in postnotes to use a colon not comma
\renewcommand{\postnotedelim}{\iffieldpages{postnote}{\addcolon\space}{\addcomma\space}} 
\DeclareFieldFormat{postnote}{#1} 

%Fix date format in bib -- could be moved to external .lbx file (with comments; change DefineBib... to DeclareBib...)
%%%%\DeclareLanguageMapping{american}{american-dmy}
%%%%
%%%%\begin{filecontents}{american-dmy.lbx}
%%%%\ProvidesFile{american-dmy.lbx}[american localisation with dmydate format for long dates]
%%%%
%%%%\InheritBibliographyExtras{american}
\DefineBibliographyExtras{american}{%
  \protected\def\mkbibdatelong#1#2#3{%
    \iffieldundef{#3}
      {}
      {\stripzeros{\thefield{#3}}%
       \iffieldundef{#2}{}{\nobreakspace}}%
    \iffieldundef{#2}
      {}
      {\mkbibmonth{\thefield{#2}}%
       \iffieldundef{#1}{}{\space}}%
    \iffieldbibstring{#1}{\bibstring{\thefield{#1}}}{\stripzeros{\thefield{#1}}}}%
}
%%%%\InheritBibliographyStrings{american}
%%%%\endinput
%%%%\end{filecontents}

% *************** Make ref links whole ref in text *****************
%%%%\DeclareCiteCommand{\cite}
%%%%  {\usebibmacro{prenote}}
%%%%  {\usebibmacro{citeindex}%
%%%%   \printtext[bibhyperref]{\usebibmacro{cite}}}
%%%%  {\multicitedelim}
%%%%  {\usebibmacro{postnote}}
%%%%
%%%%\DeclareCiteCommand*{\cite}
%%%%  {\usebibmacro{prenote}}
%%%%  {\usebibmacro{citeindex}%
%%%%   \printtext[bibhyperref]{\usebibmacro{citeyear}}}
%%%%  {\multicitedelim}
%%%%  {\usebibmacro{postnote}}
%%%%
%%%%\DeclareCiteCommand{\parencite}[\mkbibparens]
%%%%  {\usebibmacro{prenote}}
%%%%  {\usebibmacro{citeindex}%
%%%%    \printtext[bibhyperref]{\usebibmacro{cite}}}
%%%%  {\multicitedelim}
%%%%  {\usebibmacro{postnote}}
%%%%
%%%%\DeclareCiteCommand*{\parencite}[\mkbibparens]
%%%%  {\usebibmacro{prenote}}
%%%%  {\usebibmacro{citeindex}%
%%%%    \printtext[bibhyperref]{\usebibmacro{citeyear}}}
%%%%  {\multicitedelim}
%%%%  {\usebibmacro{postnote}}
%%%%
%%%%\DeclareCiteCommand{\footcite}[\mkbibfootnote]
%%%%  {\usebibmacro{prenote}}
%%%%  {\usebibmacro{citeindex}%
%%%%  \printtext[bibhyperref]{ \usebibmacro{cite}}}
%%%%  {\multicitedelim}
%%%%  {\usebibmacro{postnote}}
%%%%
%%%%\DeclareCiteCommand{\footcitetext}[\mkbibfootnotetext]
%%%%  {\usebibmacro{prenote}}
%%%%  {\usebibmacro{citeindex}%
%%%%   \printtext[bibhyperref]{\usebibmacro{cite}}}
%%%%  {\multicitedelim}
%%%%  {\usebibmacro{postnote}}
%%%%
%%%%\DeclareCiteCommand{\textcite}
%%%%  {\boolfalse{cbx:parens}}
%%%%  {\usebibmacro{citeindex}%
%%%%   \printtext[bibhyperref]{\usebibmacro{textcite}}}
%%%%  {\ifbool{cbx:parens}
%%%%     {\bibcloseparen\global\boolfalse{cbx:parens}}
%%%%     {}%
%%%%   \multicitedelim}
%%%%  {\usebibmacro{textcite:postnote}}
  
%Set custom strings for url and date labels (available at and last accessed)
\DefineBibliographyStrings{american}{%
  url = {available at},
  urlseen = {last accessed}, 
}
%\DeclareFieldFormat{url}{\bibstring{url}\addcolon\space\url{#1}}  % removed "Available at:" 2014-09-01
\DeclareFieldFormat{urldate}{\mkbibparens{\bibstring{urlseen}\space{#1}}}

% Set url + date appearance in bib (add doi here if needed)
\renewbibmacro*{bib+doi+url}{%
  \usebibmacro{url+urldate}
  }
  

% Abbreviate edited by as eds.
\DeclareFieldFormat{editortype}{\mkbibparens{#1}}
  
%Remove Quotes around titles
\DeclareFieldFormat
  [article,inbook,incollection,inproceedings,patent,thesis,unpublished]
  {title}{#1\isdot}
  
\appto{\bibsetup}{\raggedright} %align left to prevent stretched urls (perhaps remove?)

% /end bib settings



% *************** Page layout ***************
%\settypeblocksize{*}{32pc}{1.618}

%\raggedbottom  % turn of vertical justification
%\raggedright  % turn of horizontal justification

\setlrmarginsandblock{1.5in}{1.0in}{*}  % Left and right margins
\setulmarginsandblock{1.5in}{1.0in}{*}  % Top and bottom margins
\checkandfixthelayout

\setheadfoot{\onelineskip}{2\onelineskip}
%\setheaderspaces{*}{2\onelineskip}{*}

\def\baselinestretch{2}  % double space for PSU

\checkandfixthelayout


% *************** Chapter and section style ***************
\makechapterstyle{mychapterstyle}{%
    \renewcommand{\chapnamefont}{\normalfont\sffamily\bfseries}%\large}%
    \renewcommand{\chapnumfont}{\normalfont\sffamily\bfseries}%\large}%
    \renewcommand{\chaptitlefont}{\normalfont\sffamily\bfseries}%\LARGE}%
    \renewcommand{\printchaptertitle}[1]{%
        \chaptitlefont{##1}
        }%
    \renewcommand{\printchapternum}{%
        \chapnumfont\thechapter%
        }%
       
}

\renewcommand*{\cftappendixname}{Appendix\space}  % Table of contents to show Appendix A instead of just A


\chapterstyle{mychapterstyle}

\setsecheadstyle{\normalfont\sffamily\bfseries}%\large}
\setsubsecheadstyle{\normalfont\sffamily\mdseries}
\setsubsubsecheadstyle{\normalfont\sffamily\mdseries}
\setparaheadstyle{\normalfont\sffamily}

\nouppercaseheads  % prevents the headers from being uppercase
\makeevenhead{headings}{\normalfont\sffamily\mdseries\footnotesize\rightmark\small\ }{}{\normalfont\sffamily\mdseries\small\thepage}  % the "\small \ " is to vertically align the left and right parts of the header because of the difference in font size
\makeoddhead{headings}{\normalfont\sffamily\mdseries\footnotesize\rightmark\small\ }{}{\normalfont\sffamily\mdseries\small\thepage}

\makepagestyle{chapfirst}
\makeoddhead{chapfirst}{}{}{\normalfont\sffamily\mdseries\small\thepage}
\makeevenhead{chapfirst}{}{}{\normalfont\sffamily\mdseries\small\thepage}

%\aliaspagestyle{chapter}{empty}  % this suppresses numbers on chapters

\aliaspagestyle{chapter}{chapfirst}

\usepackage{indentfirst}  % indent first paragraph


% *************** Table of contents style ***************
\settocdepth{subsubsection}

\setsecnumdepth{subsubsection}
\maxsecnumdepth{subsubsection}
\settocdepth{subsubsection}
\maxtocdepth{subsubsection}


% ********** Commands for epigraphs **********
\setlength{\epigraphwidth}{0.57\textwidth}
\setlength{\epigraphrule}{0pt}
\setlength{\beforeepigraphskip}{1\baselineskip}
\setlength{\afterepigraphskip}{2\baselineskip}

\newcommand{\epitext}{\sffamily\itshape}
\newcommand{\epiauthor}{\sffamily\scshape ---~}
\newcommand{\epititle}{\sffamily\itshape}
\newcommand{\epidate}{\sffamily\scshape}
\newcommand{\episkip}{\medskip}

\newcommand{\myepigraph}[4]{%
	\epigraph{\epitext #1\episkip}{\epiauthor #2\\\epititle #3 \epidate(#4)}\noindent}
	
	
% **************** Blank Page Commmand ***************
\usepackage{afterpage}
\newcommand\blankpage{%
    \null
    \thispagestyle{empty}%
    %\addtocounter{page}{-1}%
    \newpage}
    

% *************** DRAFT OPTIONS ****************
\usepackage{ifdraft}
%USAGE:
%\ifdraft{<draft case>}{<final case>}
%\ifoptiondraft {⟨option draft given⟩} {⟨option draft not given⟩}
%\ifoptionfinal {⟨option final given⟩} {⟨option final not given⟩}

% *************** Enable hyperlinks in PDF documents ***************
%\ifpdf
%    \pdfcompresslevel=9
%        \usepackage[plainpages=false,pdfpagelabels,bookmarksnumbered,%
%        colorlinks=true,%
%        linkcolor=sepia,%
%        citecolor=sepia,%
%        filecolor=maroon,%
%        %pagecolor=red,%
%        urlcolor=sepia,%
%        pdftex,%
%        unicode]{hyperref} 
%    \pdfimageresolution=600
%    \usepackage{thumbpdf} 
%\else
%    \usepackage{hyperref}
%\fi

\usepackage[hidelinks, draft=false]{hyperref}
%\hypersetup{final}
\usepackage[noabbrev, nameinlink, capitalise]{cleveref}  % capitalise option

\usepackage[all]{hypcap}  % fixes problem when you click on a link and go to caption, not figure or table itself

%links for endnotes
%\usepackage{hyperendnotes}  % .sty is buggy and breaks document

\usepackage{memhfixc}


%***************** NEW ENVIRONMENTS *****************

\newenvironment{LastLineCentered}%
  {\setlength{\parindent}{0pt}\setlength{\leftskip}{0pt plus 0.5fil}\setlength{\rightskip}{0pt plus -0.5fil}}{\par}
  
\newenvironment{ssfigure}{%
  \begin{figure}
  \begin{Spacing}{1.2}}{%
  \end{Spacing}
  \medskip
  \end{figure}
}

\newenvironment{sstable}{%
  \begin{table}
  \begin{Spacing}{1.0}}{%
  \end{Spacing}
  \medskip
  \end{table}
}


% *************** End of document style definition ***************