\chapter{Timeline}

I hope to have finished the classification algorithm by the beginning of winter term 2014.  At that point, I will run the testing on the Kansas sample areas to find the best reference curves. Once that testing is completed, I will use those curves to classify MODIS data of Pellegrini from the 2012 -- 2013 growing season. This will provide me with a rough picture of the crop cover my method will find for 2013 -- 2014, and from this I can use a stratified random sampling technique to better ensure my ground truthing will contain sufficient sample points in each land cover class. In March 2014 I plan to do fieldwork in Pellegrini to collect the ground truth data for these points. I have chosen the month of March specifically because it is in the middle of the summer growing season, and I can get ground truth data for summer crop identification as the crops are growing. Winter crops, such as winter wheat, will need to be verified by inquiring with field owners, or confirmed by visual interpretation of Landsat imagery. In spring term 2014, I will classify 2013 -- 2014 data for Pellegrini, calculate the accuracy using the ground truth acquired in March, and begin writing. I plan to finish my work and writing over summer 2014 and present fall 2014.
