\chapter{Study Areas}

This study will use agricultural areas in Kansas, USA for testing and verification of the phenological classification method and will apply the classification method to Pellegrini, Santiago del Estero, Argentina to test its effectiveness in subtropical South America.

\section{Kansas, USA}

The state of Kansas is one of the big agricultural producers of the US. As one of the plains states, it is relatively flat across much of its extent, making it well suited to large highly-mechanized agro-industrial operations. In 2012, the three most extensive crops in the state were wheat, corn, and soybeans (Table \ref{table:kansas}), which are also the most abundant crops in Pellegrini, Argentina. Additionally, Kansas has been the focus of a number of previous studies into the use of MODIS time-series for crop classification \autocites{wardlow2002discriminating}{wardlow2005state-level}{wardlow2007analysis}{wardlow2008large-area}, and has a very detailed and easily-accessible crop cover dataset in the form of the USDA CDL, making it a natural choice for a preliminary study area to test my method.

I have chosen a small 100 MODIS pixel by 100 MODIS pixel study area just northwest of Wichita, Kansas, which includes the communities of Valley Center, Sedgwick, and Halstead running roughly in a line from the southeast corner to the northwest corner (Figure \ref{fig:KSstudysite}. This is the area where I did testing and verification of the method (see Appendix \ref{appendix:testing} for a full overview of the Kansas testing), and where I extracted the crop reference signatures used for the Argentina classification. The typical planting dates for a variety of crops in this region are shown in Table \ref{table:KSplantingdates}. This specific study site was chosen due to the good mix of land covers including, but not limited to, corn, soy, sorghum, winter wheat, winter wheat and soy double crop, urban development, grassland, and forest in the CDL reference.


\begin{sstable}
  \centering
  \caption[Most extensive crops in Kansas, 2012.]{Most extensive crops in Kansas, 2012\\~\autocite[adapted from][]{usda2013kansascrops}.}
  \label{table:kansas}
  \begin{tabu}{lcc}
    \toprule
    \textbf{Crop} & \textbf{Acreage (1,000 acres)} & \textbf{Production (1,000 units)} \\
    \midrule
    Wheat & 9,100 & 382,200 \\
    Corn & 3,950 & 379,200 \\
    Soy & 3,810 & 83,820 \\
    All Hay & 2,750 & 4,340 \\
    All Forage & 2,750 & 4,545 \\
    Sorghum & 2,100 & 81,900 \\      
    \bottomrule
  \end{tabu}
\end{sstable}

\mymissingfigure{fig:KSstudysite}{Map of Kansas study area with city/town polygons and Landsat image for background}

\begin{sstable}
  \centering
  \caption[Kansas Study Site Planting Dates]{Kansas Study Site Planting Dates\\~\autocite[adapted from][]{shroyer1996kansas}.}
  \label{table:KSplantingdates}
  \begin{tabu} to 4.5in {X[1,m,c]X[2,m,c]}
    \toprule
    \textbf{Crop} & \textbf{Planting Date Range} \\
    \midrule
    Wheat & September 25 to October 20 \\
    Triticale & September 1 to September 25 \\
    Winter Barley & September 15 to October 10 \\
    Spring Barley & February 25 to March 15 \\
    Spring Wheat & February 25 to March 15 \\
    Spring Oats & February 25 to March 15 \\
    Corn & April 1 to May 10 \\
    Sorghum & May 15 to June 20 \\
    Soybeans & May 5 to June 10 \\
    \bottomrule
  \end{tabu}
\end{sstable}


\section{Pellegrini, Santiago del Estero, Argentina}

Santiago del Estero, a province in Northwest Argentina, has an area of 136,351 square kilometers, about the same as Arkansas, but a population of about 874,000 \autocite{estadistica-y-c2010a}. The entire province is classified within the \textit{Parque Chaqueño} (Chaco forest), but, like the rest of Argentina forests, the forested area has declined rapidly over the past fifteen years. During the period 1998 to 2002, 306,055 hectares were deforested \autocite{secretaria-de-a2007informe}. From 2006 through 2011, a further 701,030 hectares of forest were lost, 283,669 of which were after the enacting of the OTBN \autocite{secreteria-de-a2012monitoreo}. Over both of these time periods Santiago del Estero experienced the highest levels of deforestation in all of Argentina.

The Department of Pellegrini is an administrative area in the Northwest corner of Santiago del Estero (Fig. \ref{fig:pellegrini}).\footnote{The Pellegrini boundary shapefile I obtained does not accurately reflect the bounds of the department on the ground. Particularly along the lengthy and straight northwestern edge, careful inspection of Figure \ref{fig:pellegrini} reveals a lack of registration between the vector geometry and the obvious boundary visible in the background image. When investigating some of my sample points along the northern and southern edges, I got strange looks and comments about how this or that field was not within Pellegrini, my supposed study area. So with this note, I want to acknowledge that I realize my study area is not actually the Department of Pellegrini proper, but an inaccurate representation as defined by the shapefiles from the Internet. I use this inaccurate representation to ensure consistency, to allow repeatability, and to simplify spatial analysis.} The department has an area of 6,944 square kilometers, a size slightly larger than the state of Delaware, and a 2010 population of only 20,514 \autocite{estadistica-y-c2010b}. The primary municipality of the department is Nueva Esperanza, with a population of about 4,500. The frontier nature of Pellegrini seems to have limited deforestation in the department for some time, but the push for land has increased the rate of deforestation. Over the years 2001 to 2005, only 5,968 hectares were found to be deforested (Volante 2005). From 2006 to 2011 the area deforested increased to 75,349 hectares, some 39,480 hectares cut after the enacting of the OTBN, a rate much higher than previously witnessed \autocite{secreteria-de-a2012monitoreo}. Of the area cleared post-OTBN, 2,181 hectares were in red areas, the highest clearing of that designation in the nation. The vast majority of clearing, however, was 29,796 hectares in yellow areas. While Pellegrini’s total deforestation during the period 2006 to 2011 was not the highest in Santiago del Estero, as both Moreno Department and Alberdi Department had higher total deforestation, as a percent of total land area Pellegrini’s deforestation occurred at a greater rate: 10.85 percent of Pellegrini’s land area was cleared versus 10.45 percent and 7.91 percent of Moreno and Alberdi, respectively.

\textcite{volante2005analisis} found Pellegrini's primary summer crop over the years 2000 to 2005 to be soy, averaging about 40,000 hectares cultivated per year. Corn was the second most frequent crop, occupying about 7,500 hectares per year. \textit{Poroto}, a generic term for many types of common beans, were the third most popular, averaging a total cultivation of about 2,500 hectares per year. The primary winter crop was wheat, though cultivation varied wildly from less than 10,000 hectares in 2002 to over 31,000 hectares in 2004.
