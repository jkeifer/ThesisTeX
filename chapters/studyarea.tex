\chapter{Study Areas}

This study will use agricultural areas in Kansas, USA for testing and verification of the phenological classification method and will apply the classification method to Pellegrini, Santiago del Estero, Argentina to test its effectiveness in subtropical South America.

\section{Kansas, USA}

The state of Kansas is one of the big agricultural producers of the US. As one of the plains states, it is relatively flat across much of its extent, making it well suited to large highly-mechanized agro-industrial operations. In 2012, the three most extensive crops in the state were wheat, corn, and soybeans (Table \ref{table:kansas}), which are also the most abundant crops in Pellegrini, Argentina. Additionally, Kansas has been the focus of a number of previous studies into the use of MODIS time-series for crop classification \autocites{wardlow2002discriminating}{wardlow2005state-level}{wardlow2007analysis}{wardlow2008large-area}, and has a very detailed and easily-accessible crop cover dataset in the form of the USDA CDL, making it a natural choice for a preliminary study area to test my method.

\begin{table}
  \centering
  \caption{Most extensive crops in Kansas, 2012}%\\\autocite[adapted from][]{usda2013kansascrops}.}
  \label{table:kansas}
  \begin{tabular}{lcc}
    \toprule
     & Acreage (1,000 acres) & Production (1,000 units) \\
    \midrule
    Wheat & 9,100 & 382,200 \\
    Corn & 3,950 & 379,200 \\
    Soy & 3,810 & 83,820 \\
    All Hay & 2,750 & 4,340 \\
    All Forage & 2,750 & 4,545 \\
    Sorghum & 2,100 & 81,900 \\      
    \bottomrule
  \end{tabular}
\end{table}


\section{Pellegrini, Santiago del Estero, Argentina}

Santiago del Estero, a province in Northwest Argentina, has an area of 136,351 square kilometers, about the same as Arkansas, but a population of about 874,000 \autocite{estadistica-y-c2010a}. The entire province is classified within the \textit{Parque Chaqueño} (Chaco forest), but the forested area has declined rapidly in the past fifteen years. Over the period 1998 to 2002, 306,055 hectares were deforested \autocite{secretaria-de-a2007informe}. From 2006 through 2011, a further 701,030 hectares of forest were lost, 283,669 of which were after the enacting of the OTBN \autocite{secreteria-de-a2012monitoreo}. Over both of these time periods Santiago del Estero experienced the highest levels of deforestation in all of Argentina.

The Department of Pellegrini is an administrative area in the Northwest corner of Santiago del Estero (Fig. \ref{fig:pellegrini}). The department has an area of 6,944 square kilometers, or slightly larger than the state of Delaware, and a 2010 population of only 20,514 \autocite{estadistica-y-c2010b}. The primary municipality of the department is Nueva Esperanza, with a population of about 4,500. The frontier nature of Pellegrini seems to have limited deforestation in the department for some time, but the push for land has increased the rate of deforestation. Over the years 2001 to 2005, only 5,968 hectares were found to be deforested (Volante 2005). From 2006 to 2011 the area deforested increased to 75,349 hectares, some 39,480 hectares cut after the enacting of the OTBN, a rate much higher than previously witnessed \autocite{secreteria-de-a2012monitoreo}. Of the area cleared post-OTBN, 2,181 hectares were in red areas, the highest clearing of that designation in the nation. The vast majority of clearing, however, was 29,796 hectares in yellow areas. While Pellegrini’s total deforestation during the period 2006 to 2011 was not the highest in Santiago del Estero, as both Moreno Department and Alberdi Department had higher total deforestation, as a percent of total land area Pellegrini’s deforestation occurred at a greater rate: 10.85 percent of Pellegrini’s land area was cleared versus 10.45 percent and 7.91 percent of Moreno and Alberdi, respectively.

\textcite{volante2005analisis} found Pellegrini's primary summer crop over the years 2000 to 2005 to be soy, averaging about 40,000 hectares cultivated per year. Corn was the second most frequent crop, occupying about 7,500 hectares per year. Kidney beans were the third most popular, averaging a total cultivation of about 2,500 hectares per year. The primary winter crop was wheat, though cultivation varied wildly from less than 10,000 hectares in 2002 to over 31,000 hectares in 2004.
