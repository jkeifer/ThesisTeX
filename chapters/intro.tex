\chapter{Introduction}

Deforestation has long been a concern throughout tropical South America. However, this process of land use/land cover (LULC) change from forest to other uses has been increasingly recognized in subtropical South America as a significant source of environmental degradation. Understanding the complex dynamics of subtropical deforestation is crucial given the prominent role of forests in debates about climate change, conservation, and the protection of endangered species \autocites{geist2002proximate}{zak2004do-subtropical}{bonnie2000counting}{houghton1994the-worldwide}{sala2000global}.

Currently, many perceive growing demand for agricultural land---particularly land for soybeans---to be one of the greatest pressures on South American subtropical forests \autocites{pengue2005transgenic}{grau2005agriculture}{altieri2006gm-soybean:}. Remote sensing has given researchers a tool to classify land cover and measure deforestation, but the often used multi-spectral or multi-temporal image classification techniques require extensive ground truth information for the accurate classification of common crop types using widely-available data. Therefore, getting a complete picture of the dynamics of deforestation, including an understanding of agricultural pressures on forests, requires rarely-available high spatial or high spectral resolution data \autocite{senay2000using} or expensive field time gathering training site data. The development of a tool that can efficiently and effectively extract crop types using widely-available imagery would be of value to the field.

The primary goal of this thesis is to develop and test a phenological classification toolset that can identify and extract crop types from a multi-date vegetation index sequence assembled using free and accessible data from the National Aeronautics and Space Administration’s (NASA) Moderate Resolution Imaging Spectroradiometer (MODIS) platform. The toolset is tested in Kansas, USA using the U.S. Department of Agriculture's (USDA) Cropland Data Layer (CDL) as ground truth to derive reference temporal signatures of summer crops and verify the operation of the classifier. Using the Kansas-derived reference signatures, imagery of the 2014 summer growing season in the Department of Pellegrini, Santiago del Estero, Argentina (Fig. \ref{fig:pellegrini}) is classified to examine the toolset's applicability in subtropical South America.

The body of this thesis is broken into seven chapters. The first is this introduction. The second provides background information on deforestation and soybean cultivation in Argentina. The third introduces the two study areas where the processing methods, presented in Chapter \ref{chapter:methods}, are applied. Chapter \ref{chapter:results} reviews the results of said methods, and chapter six is a discussion of their significance. Chapter seven closes the thesis with some concluding remarks.

An additional five appendices contain supporting information. Appendix \ref{appendix:fieldwork} is a reflection of my time in Argentina doing the fieldwork for this project. Appendix \ref{appendix:tools} documents the toolset created to implement the processing methods of Chapter \ref{chapter:methods} and the development process. All the initial testing of the tools is reviewed in Appendix \ref{appendix:testing}. Appendix \ref{appendix:future} contains a collection of ideas for future testing and research. Lastly, Appendix \ref{appendix:cdl} is a brief note about the CDL dataset.

\begin{figure}
  \centering
  \includegraphics[width=.7\textwidth]{Graphics/pellegrini2.png}
  \caption{The department of Pellegrini, Santiago Del Estero, Argentina.}
  \label{fig:pellegrini}
\end{figure}
