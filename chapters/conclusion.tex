\chapter{Conclusion}
\label{conclusion}

% Address that the method CAN create a classification without ground truth, but it is difficult to ensure that said classification is the best one, as the tool to iterate through thresholds requires ground truth for the accuracy assessment. I do believe, however, that future research may be able to work around this issue by devising a method which can use the statistics of the RMSE rasters to determine good threshold values. 

Deforestation in Argentina continues despite regulations intended to put an end to the loss of native forests. Popular perception holds the rapid expansion of soybeans throughout the country responsible for the rush on agricultural land and consequent pressure on forests. Because of this, I wanted to develop a remote sensing toolset that would allow better study of agricultural crops in the region by classifying crops by specific types.

I believe I have successfully developed such a toolset. However, my results do not show that it can be used effectively to classify crop types globally and under any conditions, as I had originally hoped.

Agricultural practices are key assumptions in the design of any crop classifier. The importance of a complete understanding of agricultural practices in the study area must not be underestimated. Such an understanding allows one to choose an appropriate classification method given the growing conditions. When differences in crop planting and management violate the classifier's assumptions, the classifier and all other similarly-assuming classifiers break down. In the case of my work in Pellegrini, the assumptions of my classifier did not match the growing conditions, and the classifier was unable to differentiate between the summer crops I wished to identify.

Rigorous testing and optimization of the classification tools and workflow developed in this study may eliminate some of the current assumptions. I believe such work can only improve classification accuracies and the conditions under which this classifier is useful.  Many ideas for testing and future research are presented in \autoref{appendix:future} in the hope that other researchers will continue this investigation. I am encouraged by the results of this study and see these methods to hold promise for those investigating agriculture and land use and land cover change. Having said this, I must admit my recognition of the importance of this work has changed somewhat.

I initially thought that classifying specific crops is necessary to understanding the full dynamics of deforestation. While I still believe that to be true, my perception has shifted because if my experience in Argentina, and the data I have gathered. Knowing which crops are highest in demand and expanding most rapidly is often enough to get a picture of the pressures driving agricultural expansion; knowing what is being grown in a field in any given year is not as important as I thought. Why? Farmers can change their crops annually, so where soy is occurring one year, corn or sorghum or some other crop entirely may be grown the next, as the shift from soy to corn in Pellegrini may demonstrate. Nonetheless, accurate land-cover maps generated by remote sensing methods could be used for LULC change studies and modeling that would provide more comprehensive insights into the actual drivers of LULC change.

Moreover, I have an newfound understanding of the influence of economics in LULC change. Popular wisdom in Argentina recommends two investments for securing wealth: buy dollars, or buy land. Correspondingly, agricultural land prices are extremely high \autocite{mercopress2010prime}; I was quoted \$20,000 USD per hectare for land around Nueva Esperanza. Prices in the more productive Pampas region are higher still. Anecdotally, one Pellegrini land owner, who resides in Rosario, told me he bought 3,000 hectares in Pellegrini for the same price fifty hectares of land surrounding Rosario was selling for.

Even so, many large parcels in the area were for sale; I cannot count the number of times I was offered a ``good deal'' by some land owner looking to cash in and get out. I was left with the impression that the deforestation, the mad land grab, is not about agriculture. It is about desperately securing wealth in the unstable Argentine economy, where inflation runs rampant and a shadow economy rules all. It is an investment bubble. And it just so happens to be threatening the ecological health of the country, and the planet as a whole. Perhaps it will burst, and the cutting will cease. But by then will it be too late? Will a consequent fall in production simply begin to inflate a new bubble somewhere else?











