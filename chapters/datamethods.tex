\chapter{Data and Methods}
\label{methods}

\section{Overview}

This study has two purposes: (1) develop a set of tools to allow the classification of agricultural crops using a time series of imagery and known crop reference signatures, and (2) test the portability of said reference signatures. The known crop reference signatures were extracted from the Kansas study site, and their portability was tested by using them to classify the Argentina study site. The data used for this study consist of the following:

\begin{Spacing}{1.2}
\begin{itemize}
  \item 250-meter MODIS 16-day Composite Vegetation Index images
  \item 30-meter 2012 USDA Cropland Data Layer: agricultural land cover raster dataset
  \item 30-meter Landsat 8 Operational Land Imager satellite imagery
  \item Shapefile of the administrative boundary of the Department of Pellegrini
\end{itemize}
\end{Spacing}

The MODIS VI composites were used as the source imagery for extracting phenological signatures and classifying the crop types. The CDL served as a ground truth for the Kansas study area. This dataset allowed identification of crop clusters when extracting signatures, and validation of the classification method. Landsat 8 OLI images of Pellegrini (path 230, row 78) from multiple dates were used during field data collection and subsequent digitizing. All three of these data are publicly available from U.S. agencies. The Pellegrini boundary, used to define the Argentina study area, is from the GDAM Global Administrative Areas Dataset, version 1.0 (available at \url{http://www.gdam.org}).

Additionally, for the Pellegrini classification, I used a 2014 Pellegrini Land Cover vector dataset with crop identifications. I produced this dataset as a result of my fieldwork. As mentioned above, the data were digitized from Landsat 8 images, and land cover identifications were collected in the field.

An outline of the processing workflow is below:

\begin{Spacing}{1.2}
\begin{enumerate}
  \item Reproject the MODIS composite imagery.
  \item Assemble individual composite images into time series images: one for Kansas and one for Argentina.
  \item Identify all the ``pure'' pixels---those containing a single land cover type (e.g.\ non-mixels)---in the Kansas TSI by overlaying the CDL with the MODIS pixel grid.
  \item Use the CDL to isolate the pure corn, soy, and sorghum pixels in the Kansas TSI.
  \item Identify the unique phenological groups in each set of isolated Kansas TSI pixels using k-means clustering.
  \item Extract the pixel values for each unique phenological group from the Kansas TSI and find the mean value for each date to form the unique signatures for each crop.
  \item Fit the Kansas signatures to the Kansas TSI using the TSF method from \autoref{methods:phenology-fitting}.
  \item Classify the Kansas RMSE rasters and assess the classification accuracy.
  \item Fit the Kansas signatures to the Argentina TSI.
  \item Classify the Argentina RMSE rasters and assess the accuracy.
\end{enumerate}
\end{Spacing}

This chapter is a look at the methods and fieldwork used to create the validation land cover dataset of Pellegrini, and the data processing steps used to generate the study results. Details about the specific tools in the classification toolset and the development process can be found in \autoref{appendix:tools}. For a detailed explanation of all the testing proceeding this study, please see \autoref{appendix:testing}. A thorough recounting of my field experience can be found in \autoref{appendix:fieldwork}.


\section{Field Methods and Data Collection in Pellegrini}

While ground truth data were easily obtained for the Kansas study site, getting a ground truth dataset for verification of the classification in Pellegrini was not so simple. Such a dataset did not exist, necessitating onsite data collection. I visited Argentina during the summer 2014 growing season to gather field observations of summer crop types and to talk to local farmers about typical agricultural practices, summer and winter crop varieties, and planting and harvesting dates.

To guide my ground truth collection, I generated 400 random sample points inside the Pellegrini shapefile boundary, and used a Landsat OLI image as a reference for land cover (image date \formatdate{5}{2}{2014}). Where a point fell within a MODIS mixel, I allowed it to be moved within a 3-by-3 pixel window centered on the point's original pixel. As much as possible, I aimed to keep the point within a pixel belonging to the feature type on which it originally fell. In certain limited cases, if a point fell quite obviously within a field but the center pixel and eight surrounding pixels were mixels and/or were not within the field, I allowed the point to be moved to the closest pure pixel in the same field. Of the 400 sample points, I had to move 106 within the 3-by-3 window, and ten to a non-neighboring pixel within the same field.

From these adjusted points, I was able to establish 247 were forested using Landsat imagery. I did not need to visit these. For each of the remaining 153, I created map sheets---one for every point, centered on the point---showing the point on satellite imagery at three different scales: (1) an overview at 1:60,000 scale with the \formatdate{5}{2}{2014} Landsat OLI image; (2) a closer view at 1:30,000 scale, with the Landsat OLI image and an overlay of the MODIS pixel grid; and (3) a large scale 1:4,500 view with older but higher spatial resolution imagery from Digital Globe. I also made eighteen smaller-scale ``regional'' maps at 1:150,000 scale from a 25-kilometer grid overlaid on Pellegrini, which I used to identify neighboring points and plan routes. Lastly, I made an overview map of the entire department at 1:475,000 scale. I printed all of these maps and put them in a binder, so I could take notes and record the crop types in the field. I also planned to collect data about fields without sample points; I wanted to record every field I came across to ensure the comprehensiveness of my ground truth dataset.
 
The primary means of gathering the crop identities for the ground truth was direct observation. When direct observation was not possible, I interviewed local farmers and land owners. In such cases, the interviewees were asked to identify their fields on the printed maps and describe their cultivars. Finally, if no knowledgable local could be found, visual interpretation of the Landsat imagery was attempted using the appearance of already-identified fields as reference. If an identification was uncertain, the point was removed from the sample set. All the collected data were recorded directly on the sample point maps and later manually digitized into a geographic information system (GIS).

Ancillary information about agricultural practices in the region was also collected whenever possible. It was a key goal to identify each crop's date range for planting and harvesting in order to allow: (1) the proper selection of MODIS imagery dates, and, (2) appropriately setting the $tshift$ bounds for \autoref{eq:gofx}, enabling the temporal alignment of the Kansas reference signatures with the date range of the Pellegrini growing season.

\section{Data Processing}

\subsection{Resampling the CDL}

To use the CDL as a ground truth with the Kansas TSI, the 30-meter CDL pixels were resampled by majority to match the larger TSI pixels. This allowed a direct comparison between the crop values from the CDL and the pixel signatures in the TSI.

\subsection{Building the TSIs}
\label{buildingTSIs}

For this study, I chose to classify the 2012 Kansas summer growing season and the 2014\footnote{To clarify, I have chosen to refer to the summer growing season by the harvesting year, given that the Argentine summer technically ranges from \datenoyear{21}{12} to \datenoyear{20}{3}. This is the same date range as the Northern Hemisphere's winter season, which is typically referred to by the year in which it ends, i.e.,\ winter 2014. I find other references to the growing seasons in Argentina to be somewhat confusing. Often, a year range is used, such as 2013-2014, to denote that the growing season actually begins in 2013. Sometimes, as planting dates in Argentina as a whole are primarily September through December, the proceeding year is used to refer to the entire growing season, which I find especially misleading.} Argentina summer growing season. I classified only summer crops instead of a full agricultural year because I lacked ground truth data for winter crops in Pellegrini.  Both NDVI and EVI preprocessed data are available as MODIS products, but I did not test the latter. This decision was based on preliminary test results: NDVI outperformed EVI in this type of classification. For more information on this testing see \autoref{appendix:testing:r1} on \autopageref{appendix:testing:r1}.

I assembled the MODIS 16-day composite NDVI images into multi-date time-series images (TSI) covering the growth cycle of the summer crops. Each band in the TSIs is a 16-day composite NDVI image and the bands are ordered consecutively (see \autoref{appendix:tools:build} on \autopageref{appendix:tools:build} for the description of the Build Multidate Image Tool). The Kansas summer TSI covered the date range DOY 97 through DOY 273, and was made with data from the Terra satellite (LPDAAC product MOD13Q1, tile h10v05). Prior to creating the TSIs, each of the 16-day composites was reprojected from the native MODIS sinusoidal reference system using LPDAAC's MODIS Reprojection Tool \autocite{modis4.1}. I reprojected the Kansas data into the Albers Equal Area Conic projection for the contiguous U.S. using the 1983 North American Datum (WKID: 5070) to match the reference system of the USDA CDL.

In Argentina, as it is in the Southern Hemisphere and the seasons are inverted to those of the Northern hemisphere, the growing season shifts, as must the date range for the VI time-series. The TSI for Pellegrini must begin at the end of the proceeding year to adequately capture the entirety of the summer phenologies. To accomplish this, the time-series image for summer 2014 began with the 16-day composite image from DOY 353 of 2013 (or DOY −13 with reference to 2014) and ended with the image from DOY 161 of 2014 (MODIS grid tile h12v11). This specific date range was chosen based on information provided by local farmers to ensure coverage of the earliest planting and latest harvesting dates, as well as manual inspection of crop pixel signatures throughout the study area. Persistent clouds necessitated using an image from the Aqua satellite of DOY 105 (LPDAAC product MYD13Q1) in place of DOY 113 from Terra. Due to no viable imagery from Aqua or Terra, DOY 129 was interpolated via the arithmetic mean calculated from the DOY 105 Aqua image and the DOY 145 Terra image. The Argentina composite images were reprojected to the UTM Zone 20S reference system (WKID: 32720).

\subsection{Eliminating Mixels}
\label{methods:mixels}

After building the TSIs, the next processing step was to eliminate mixels from the Kansas TSI. Mixels, or pixels that contain more than one land cover type within the 250-by-250-meter cell area, have mixed temporal signatures. Earlier testing showed that mixels can not be clearly classified (see \autoref{appendix:testing:r2} on \autopageref{appendix:testing:r2} for more information). To prevent errors, mixels had to be removed from the Kansas analysis. Pure pixels, the non-mixels in an image, were isolated using ArcGIS \autocite{arcgis10.1} by intersecting a vectorized version of the original 30-meter CDL ground truth with a vector grid of the TSI pixels. From the resulting geometry, all polygons with an area greater than 53,000 square meters (or 98 percent of a MODIS pixel) were selected as pure. I also manually selected two sorghum pixel features that were omitted due to intermixed soy pixels. I chose to add these pixels due to the low number of sorghum features retained, and because the CDL's intermixed soy pixels appeared to be errors.

All of the other polygon features were discarded. The remaining polygon features were converted to point features based on their centroids. These points were used to find the TSI pixel coordinates of the pure pixels in the subsequent processing steps, including extracting the reference signatures. 

\subsection{Extracting the Reference Temporal Signatures}
\label{methods:sigextraction}

As a reference library of temporal signatures does not yet exist, I had to extract my own signatures from the Kansas TSI. To do so, the pure TSI pixels of each key summer crop (corn, soy, and sorghum) as specified in the resampled CDL were isolated in separate rasters. Initial classification attempts revealed high variability in the crop temporal signatures. I obtained the best classification results when using more than one signature per crop. Given the existing literature about phenological classification, I do not actually believe crops have multiple signatures. I believe every crop as a theoretical ideal signature, which can be made to fit any actual signature using the $xscale$, $yscale$, and $tshift$ transformations in \autoref{eq:gofx}, barring any unusual effects of weather or other forces impacting crop development (a mid-season drought, for example, may cause a crop signature with double peaks due to the partial dying off and regeneration of the plants). However, given the CDL as my only source of ground truth, I had to assume it was accurate, despite any evidence to the contrary. Therefore, I strived to create a classification with the highest accuracy compared to the CDL.

To extract multiple signatures for each crop, the separated rasters were each clustered into three clusters using the ENVI \autocite{envi5.0} k-means tool. K-means is an iterative unsupervised classification technique where data are clustered using the minimum distance to the nearest cluster centroid \autocites{kmeans2014}{matteucciclustering:}. The centroids begin equally distributed throughout the data, but are recalculated on each iteration based on the shapes of the clusters in the previous iteration. The algorithm stops after the maximum number of iterations have been run, or when the cluster pixel counts change less than a given percent threshold. I chose to use a 1.0 percent change threshold over 100 maximum iterations. The TSI pixels in each cluster were then sampled and averaged. That is, the values from every pixel in a cluster on a given date were averaged together, providing the reference signature value for that date. This averaging is repeated for all dates in the TSI, and the result is the complete reference temporal signature representing that cluster (an illustration of an average signature from pixel signatures is shown in \autoref{fig:averaging}). Completion of this process for all crop clusters creates the three primary signatures for each crop (see \autoref{appendix:tools:extract} on \autopageref{appendix:tools:extract} for information about the Extract Signatures Tool).

\begin{ssfigure}
  \centering
  \input{plots/firstwheatKS.pgf}
  \caption{Five Winter Wheat Pixel Signatures and Their Mean Reference Signature}
  \label{fig:averaging}
  \medskip
  \small
  This example shows the averaging process used to derive reference temporal signatures from sampled pixels. At each date (band) in the TSI, the pixel values are extracted and averaged via their mean. The bold line shows the resulting signature created from the averages.
\end{ssfigure}


\subsection{Fitting the Reference Signatures to the TSIs and Creating the RMSE Rasters}

To find the degree of fit between reference signatures and unknown pixels, I used the TSF comparison method without the wavelet filtering step. I implemented a tool based on \autoref{eq:1} (\autopageref{eq:1}) using the programming language Python \autocite{python2.7.8}. Details of the tool and its development can be found in \autoref{appendix:tools:fit} on \autopageref{appendix:tools:fit}. The tool iterates through each pixel in a TSI, comparing each reference signature with the pixel's values (i.e.\ MODIS NDVI values) and calculating the RMSE between the signature and the pixel's values. An output RMSE raster is created for each crop signature to record the RMSE values for each pixel. I omitted the wavelet filtering step because \citeauthor{sakamoto2010a-two-step} used higher temporal resolution data than I had, and I did not think filtering would be as necessary or as effective with the 16-day interval of the MODIS VI composites I used.

The reference signatures extracted from the Kansas TSI clusters were fit to the Kansas TSI and the output RMSE rasters were classified and verified. Then, the same Kansas signatures were fit to the Pellegrini TSI within the study area. The output RMSE rasters from each crop type were then classified. For both Kansas and Pellegrini, the bounds for the $xscale$ and $yscale$ parameters, which account for differences in growing season length and plant health, respectively, were 0.6 to 1.4. The $tshift$ parameter, which temporally aligns the reference signature to a pixel's values, was bounded to -10 to +10 days for the Kansas fit processing. The $tshift$ bounds for the Pellegrini fit processing were 120 to 140 days (still $\pm10$ days, but shifted 130 days). These bounding ranges were chosen as the result of non-extensive testing. Given the different lengths of planting date ranges between the Kansas and Pellegrini sites, the $tshift$ bounds likely need to be adjusted accordingly. I am certain this is one aspect of the method that requires further testing and optimization, and could have a large effect on classification results.


\subsection{Classifying the RMSE Rasters}
\label{methods:classification}

RMSE rasters quantify the likelihood of a pixel being the crop types represented by the associated reference signatures. To create a classification from these likelihoods, one must first find the useable RMSE values: even pixels that are obviously different than a reference signature will result in measurable RMSE values, they will simply be high RMSE values. Thus, the RMSE rasters need to be thresholded to eliminate such high RMSEs.

One can think of the threshold in a similar manner to weighting: a higher RMSE threshold will allow more pixels from that RMSE raster to be considered in the final classification. The correct threshold for a RMSE raster will vary depending on a variety of factors, including what types of crops are in the sample area, what crops are trying to be identified, and how homogeneous the pixels for that crop are in comparison to the reference signature used. The extent to which these and other factors influence the optimal threshold is not well understood and requires further study. Despite not knowing the exact effects of these factors on the optimal threshold value, it is clear that the optimal value will vary between RMSE rasters. That is, a single threshold value used across all RMSE rasters in a classification is unlikely to provide the highest possible accuracy.

Once the RMSE rasters are thresholded, one can make a classification by finding the best fit for every pixel. For example, if a given pixel had thresholded RMSE values for corn, soy, and wheat of 356.7, 531.5, and none (as the original RMSE was above the threshold used), respectively, the best fitting signature would be corn, and the pixel could be classified as corn. If a pixel did not have any RMSE values under the threshold used, then that pixel would not be classified (a classification as ``other'' would be equally correct).

As mentioned, how to calculate optimal RMSE threshold values is unknown. Therefore, to find the best threshold combination, I developed a tool to brute-force through many combinations of RMSE thresholds within a user-specified range, classifying the RMSE rasters and calculating the accuracy of each threshold combination. A raster of the classification with the best accuracy is retained. More details about the tool itself can be found in \autoref{appendix:tools:classify} on \autopageref{appendix:tools:classify}. Unfortunately, this tool does require a ground truth dataset for the accuracy assessment. I am currently unable to create an optimized classification using this method without such a resource. An unoptimized classification can still be created manually without a ground truth.

For the Kansas classification, it was easy to iterate through different RMSE threshold ranges to find the highest accuracy classification. However, generating the Argentina classification was not as straightforward. As the tool must brute-force through every RMSE threshold combination, the time required to find the best threshold in a range is exponential, given by $s^n\cdot t$ where $s$ is the number of threshold steps, $n$ is the number of RMSE rasters, and $t$ the time of each iteration, in seconds. The Kansas classification took under 0.3 seconds per iteration; classifying the larger area of Pellegrini took substantially more time, somewhere between three and four seconds per iteration. The increased processing time ruled out using more than two or three threshold steps. Consequently, in the case of the Argentina classification, it was much more efficient to iterate through a range of RMSE threshold values using a single threshold value across all the RMSE rasters, then manually refine each raster's threshold value until the optimum combination was identified. That is, I found the threshold value that, when used on all the RMSE rasters, had the highest classification accuracy. Then, I manually increased or decreased the threshold value of each RMSE raster individually, until I could no longer improve the classification accuracy.