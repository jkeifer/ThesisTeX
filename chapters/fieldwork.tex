\chapter{The Story of My Field Work}
\label{appendix:fieldwork}

In order to complete an accuracy assessment of the classification I was to produce of agriculture in Pellegrini, I knew I needed ground truth data with which I could compare my results. As I suspected would be the case, I was unable to find any extant datasets, so I knew I would have to visit Pellegrini to gather such data.

After extensively reviewing satellite imagery of the area, I knew the fields were very large and appeared to have many roads connecting them, so I did not expect access to be problematic. While the size of the department, 6,943 square kilometers, is about the size of Delaware, I thought I would be able to cover ground fairly quickly, and allocated three weeks of time in Pellegrini to gather all my data.

I did have concerns to how the local people would take to my project. I know that I would be immediately suspicious of some foreigner coming in to my town and wanting to know everything about the agricultural practices in the area, including visiting all of the fields. I actually practiced how to say, in Spanish, ``Don’t shoot! I am leaving, there is no problem.'' Perhaps this is just an American thing, but I was expecting, at some point, to be confronted by someone with a gun who did not like me. After all, I am not necessarily in favor of the agriculture that is taking over the area, and while I tried to present my views as neutrally as possible, I thought a conflict would be inevitable.

I arranged for a small rental car in San Miguel de Tucumán, Tucumán, a city about 150 kilometers from Nueva Esparanza. I knew the roads would not be great, but I figured I should be able to get through just about anything with the rental car, except mud. Nueva Esparanza was to be my base, and my plan was to try and visit the furthest areas first, as I expected those to the most difficult to access, leaving the easier areas for last.

As mentioned in my methods, I randomly generated 400 points throughout the department to survey. Of those 400, I immediately identified 247 of them as forested using Landsat imagery, which meant I did not need to visit them. For the remaining 153 I did need to visit, I created map sheets—one for every point, centered on the point—showing the point at three different scales: an overview at 1:60,000 scale, a closer image at 1:30,000 scale with the MODIS pixel grid overlaid, and a very large scale 1:4,500 scale view with older but higher resolution imagery from Digital Globe. I also created a 25-kilometer grid over Pellegrini, which I used to make eighteen smaller-scale ``regional'' maps at 1:150,000 scale to help identify neighboring points and plan routes. Lastly, I made an overview map of the entire department at 1:475,000 scale. I printed all of these maps and put them in a binder. I planned to collect data about as many fields as I could, even those without sample points, and I got metallic markers so I could outline each field on the maps and take notes of crop types.

Getting to Nueva Esparanza was a challenge in itself. Due to the budget fare the airline provided me in exchange for my miles, it took me some 36 hours just to get to the hostel in Buenos Aires. Once there, I had to make my way around town to gather some supplies and change money. I had a short night in the hostel, as I needed to make an early flight from Buenos Aires to San Miguel de Tucumán. I picked up my rental car at the airport in Tucumán, at which point my stress level increased significantly, as I now had to make my way around the city not as a passenger, but as a driver. Argentine traffic laws do exist, but my perception is that, generally speaking, no one knows what they are.

Another problem was gas. Even something as simple as purchasing fuel for one’s vehicle can become a new and stressful experience in a foreign country. After visiting Guatemala, where drivers would pull up to random buildings around town and attendants would appear from nowhere with a container of gasoline and a makeshift funnel made from the top of a plastic bottle, I was unsure what to expect. It turned out that the process was not so rudimentary nor much different from buying fuel back home, and my concern was mostly unwarranted.

After driving throughout the city gathering supplies, it was time for me to head to Nueva Esparanza. Despite using two maps and my GPS to try and navigate my way to the correct highway, I found myself on the wrong road out of town, and had to spend an inordinate amount of time following a long string of slow moving cars along what seemed more to be a series of main streets through a corridor of small towns than a highway. Thankfully, however, the road eventually led to the route I initially intended to take, and I began to make more rapid progress towards Nueva Esparanza.
Unfortunately, my rapid progress quickly slowed upon reaching the beginning of road construction, which persisted the next 70 km or so.

My first full day I was in Nueva Esparanza I planned out a long route to investigate, but, after talking with some of the workers of the hotel I was staying in about my security concerns, I decided I should visit the police \textit{comisaria} to ask if I was going to have any problems with land owners or locals while working. The first moment I opened my mouth I became the attention of everyone in building. I used the best Spanish I could muster to try to tell them what I was there to do and why, but I kept getting passed from person to person. Eventually, based on the questions they were asking me—things I was sure I had already said—I realized they could not understand me. And I was struggling mightily myself to understand them.

After some two and a half hours and what must have been forty people trying to interviewing me, the police were finally able to track down an English teacher who taught in the local schools. With his assistance, I was able to communicate the details of my project, and the police made up an official document vouching for my identity and purpose in case anyone took issue with my presence in the department. Everyone assured me that I would not have any problems with the people. Not once during my trip did I need the document.

In spite of the late hour I left the police station that first day, I naïvely attempted to complete the long route I had laid out for myself. My early progress was actually quite good; I did not check off many points, but I was able to visit quite a number of fields in a short amount of time. This served only to errantly bolster my self-confidence.

In line with my plan to visit the furthest points first, I was making my way to the far northwest corner of Pellegrini. About an hour and a half from Nueva Esparanza, I came to the border with the province of Salta. Looking at the Landsat imagery, I could clearly make out a long, straight road following this line. That this road was not marked on any other maps should have been a clue to me.

After a few minutes of searching and considering the numerous side paths along the main road, I determined the correct road to follow, and proceeded to head north along it. I had a thought about the sandiness of the road, but knew that as long as I maintained my momentum I should not have any problems. I did not consider the hour, which was closing in on 5:00 PM, nor the fact that, due to my lengthly visit with the police, I had not taken the time to eat that day, aside from a breakfast consisting only of a banana.

Only a few minutes down the road, the situation quickly worsened. Deep ruts suddenly appeared in the middle of the road, and I did my best to straddle the car over the left one by attempting to drive with my left tires on the side of the road and my right tires down the middle. However, the success of this maneuver was short lived, and before I knew it, the small, low car had dropped down into the ruts. I stepped on the gas, hoping that keeping the car moving would keep me for getting stuck. The sound of the sand scraping at the underside of the car was unbearable. I spotted a break in the vegetation on the left side of the road, so I pointed the car at it, hoping the wheels would be able to break free from the ruts with a sharp enough turn. Luckily, this time my maneuvering was successful, and I found myself parked on the only hard patch of clear ground in visible surroundings.

I got out of the car and surveyed the situation. The deep, sandy ruts continued in both directions along the road. Being only a few minutes from the main road, and realizing the lateness of the hour, I figured the only reasonable course of action would be to return to Nueva Esparanza. I contemplated driving through the small brush along the side of the road to get back to where the ruts were shallower, but I figured my tires would have been no match to the large thorns common to so many of the Chaco’s plants. My only regress would be to attempt to drive back the way I came.

Careful of the thorniest of plants, I managed to turn my car around, orienting it in the proper direction. I knew I needed to do two things if I were to make it back to the main road without problems: go fast, and stay out of the ruts. I was able to do neither.

I stepped on the gas, but given the gearing of the Chevy and its abysmally-small power plant, quick of the line it was not. Adding to the fact that I was wholly unsuccessful at keeping out of the ruts, I managed all of three or four meters before losing all forward momentum. I tried rocking the car by repeatedly shifting between first and reverse, but only managed to get the car more firmly planted in the sand.

Thanks to the sound advice of Polo, one of my committee members, I had actually purchased a shovel the day before in Tucumán. To say that at this point the shove came in handy would be an understatement.

I proceeded to dig out all the sand underneath the car upon which it was high centered, as well as a short path both in front and behind the car, and attempted another run at freedom. I repeated these steps numerous times with no success. Each time I felt that much closer to collapse due my plummeting blood sugar.

Some time into this cycle a woman approached on a motorcycle with her two kids. I stood there, staring at her as she drew closer, hoping she would stop and tell me she would be right back with someone to help me out. Instead, she stopped and began looking at me as if amused as I proceeded to explain my predicament as best I could. She told me in the most unhopeful manner that she would send someone my way, if she found anyone who could help. I still wonder what became of that woman and her two kids.

Eventually, I became wise to my insanity, and decided I needed to try another course of action. I recognized my problem was acceleration: each time I tried to accelerate, my tires would dig into the sand, and the ruts would present themselves as that much deeper. In order to escape the situation, I needed to be able to get my car to speed before I encountered any ruts.

I quickly went to work regrading a 30 meter length of road. By this point I was on the verge of passing out, but I knew I could not take a break. Even if it had dawned on me that I could have eaten one of my cup of noodles raw (the only food I had in the car due to a misjudgment in preparation for this day trip) I could not have stopped to do so, as I needed to get myself out before dark or I would be stuck there all night.

After what I estimate to be two hours of hard work in that energy-sucking heat, I finally managed to clear what I hoped would be a long enough section of road to get me free. Getting back into my car, I resigned myself to spending the night out there, my cynicism taking hold and condemning me an attitude of hopelessness. Yet, in spite of my natural inclination for pessimism, I found my car floating over the sand, making its way toward freedom. I did not let of the accelerator until I was sure I was free of the clutches of the sand, which, looking back, actually may not have been until I was safely parked outside my hotel.

At this point I knew my whole plan was falling apart. I had planned to visit some twenty-five points that afternoon, yet I only made it to four. I thought I could depend on myself to get around, but I realized my car was woefully incapable of passing all but the most traveled of roads. Moreover, many of the “roads” I had spotted on the satellite imagery and was depending on to get to my points were not in fact roads, but private paths behind fences, gates, and rows of bushes, accessible only to those with the permission of the landowner.

Even the roads that were accessible were beyond my worst nightmare. Perhaps my definition of bad was inaccurate; even when people told me before my trip the roads would be bad, I just said I’d be fine. I’ve driven on bad roads; what could be the problem? This is not to say I did not expect to have \textit{any} problems with the roads, but to see the condition of the main roads—all potholed, rutted, sandy, and muddy—and to get stuck on my first day out, was a humbling and troubling experience.

I needed help.

The road conditions were not the only reason either. My interactions with the police officers at the comisaría should have been a clue to the difficulty I would have communicating. Argentine Spanish is particularly difficult on its own (if one has learned Mexican and Central American Spanish as I have), but the Spanish in Pellegrini is another dialect entirely. Take Argentine Spanish and add indigenous terms and the accent and idioms of an isolated rural area, and I felt like I was trying to learn another whole language. If I could have chosen one thing to have made my trip smoother, it would have been a better command of Pellegrini Spanish. I was able to get by, and came to understand some individuals fairly well, but often I found myself unable to communicate effectively.

I became clear to me that I needed to break out of my comfort zone and rely on others. Doing so was very hard for me, as I tend to be extremely independent and like to do everything myself. I set unreasonably high standards, and few, including myself, can live up to them. However, I was clearly not succeeding on my own; I needed people that could get me to the places I had to see. It turned out that the English teach just so happened to have a motorcycle, and graciously agreed to take me out to survey points in the afternoons when he had free time. He also introduced me to a local teenager, who, despite his age, proved to be a worthy guide, as he knew the area and some English. His grandfather also had a truck, which helped us get around.

In working with these guides, I quickly came to learn that I was going to have the best success not in going to every survey point myself, but in talking with local farm hands and landowners. Even with the right vehicle, many places were still inaccessible, primarily because many “roads” on the satellite imagery didn’t connect, or were blocked by locked gates. Luckily, I was introduced to one police officer in particular whose main function was to know everything and everyone in Pellegrini. Not only did he know what roads I could and couldn’t drive on in my car, but he also had a way with people that allowed him to get much more information that anyone else I worked with. By far, the connections I made through him and the subsequent interviews supplied the majority of the data I collected. One such connection was with his cousin. His cousin was not only exceptionally knowledgable about agriculture in Pellegrini and managed quite a number of fields, but actually took it upon himself to gather some of my data for me, visiting some rather remote fields and talking with a number of other producers he knew.

Trusting people—especially people I do not know—with a project as big and important as my master’s thesis took an intentional act of letting go. I had to realize I needed skills and knowledge I did not have, but those around me did. This was doubly hard considering my inadequate lingual skills, and that at times communication would break down. What’s more, I couldn’t decide who was going to help me; the people I preferred to work with were not always available, so I had to turn to others I would not necessarily have chosen. Anyone doing fieldwork must be prepared for this reality: you can’t choose the people that will be willing to help you.

I suppose I already knew I would need to do this, intellectually, but actually being able to was a learning opportunity for me. Letting go and trusting did not come easy, and often didn’t really happen; I merely internalized my uncertainty in others as stress.\footnote{Cultural differences were not helpful in this regard; everyone is extremely laid back; it felt as though few understood the time constraints of my work. However, I even found myself with such an attitude, most likely because of the effects of siestas and eating dinner well into the night; it is hard to get much done after midday. Additionally, the food was generally not my favorite. A lack of calories and sleep conspired to keep my energy levels depressed, so I was often content just to sit around and abide the relaxed atmosphere. Under other circumstances, I would have greatly appreciated this un-busyness. The constant assault of work is, I believe, a severe plight of the American culture, and the Argentine contentedness with leisure is refreshing. Under the looming pressure of a thesis, however, the inability of this environment to foster progress becomes problematic, and increased my stress levels.

Even when I was full of energy and vigor and wanted to get things done, I was often unable to do so, because of my reliance on others who were often occupied. While I felt as though I was not doing enough, I believe they felt like they and I were doing too much. Near the end of my trip, with unfinished work looming before me, I was often told that I needed to stop stressing and relax, yet the relaxing was exactly the cause of my stress! Of course they were right though, as everything was finished in time, thank in no small part to all those who worked to help me complete my project.} Yet, despite the overwhelming stress I inflicted on myself during some particularly “trusting” moments, \textit{nothing bad happened}. I got my data. I was never robbed. I was never left stranded in the middle of nowhere (except of my own doing). My car got repaired and I didn't get ripped off. I survived.

Another realization: you never know what someone might be able to offer you. That is, I found it important to talk with everyone around me. Sometimes is was merely a different perspective or insight, while other times it was information which enabled me to cross of a couple points, but I realized everyone had something to tell me. I am an introvert and not outgoing, so I tend to shy away from most people, yet I was forced to interact with everyone. Many people I honestly would have avoided under different circumstances. I even found myself doing things that made me uncomfortable just to build my credibility, such as going to the \textit{boliche} at three in the morning and trying (and failing) to dance to the popular music. Surprisingly, I ran into a couple landowners in the club that night, and I could tell their impression of me was positively affected just by me being there. Joining in the cultural customs builds a rapport better than anything else.

These activities and having to build relationships I would normally have avoided pushed me outside my comfort zone and were a great opportunity for personal growth. Reflecting back on the trip and my life since returning, I can see a greater degree of social confidence. I am still shy and introverted, but I no longer feel unable to put myself out there when meeting new people. Moreover, some relationships I might have otherwise written off turned into good friendships.

As is evident, I was overly confident in my abilities, and consequently made a number of incorrect assumptions about how my work would go. Thankfully, of all things, the data collection maps I made worked very well. If I were to have to plan such a project again, I would struggle to identify any changes I would make to them. I will say that I overestimated the usefulness of the maps for navigation; my GPS receiver with a satellite image and my sample points loaded onto it proved to work much better, as I never had to find my location on the map before identifying the next turn. The obviousness of this strikes me now; I am just grateful that I was able to download the necessary software, despite my phone’s seemingly nonexistent data connection, to make such a solution possible. Next time I will be sure to have my GPS setup beforehand.

And, to reiterate: one must trust in others. They will help. I expected them to not. Perhaps that is because their culture is more relational, or perhaps I am simply too untrusting. In any case, it is foolish to think that one can go into another culture without the expert knowledge the locals have of the place and customs\footnote{Even the simple---and apparently not universal---act of knocking on the door when visiting someone's home: what is one to do when fences, dogs, and sometimes the absence of a front door prevents knocking? Clap of course. While clapping seems an entirely logical course of action after the fact, it was not initially obvious to me, and I found my ability to collect data severely limited until I understood this, and a few other, basic rules and customs regarding social interactions.} and be able gather any data, whether those data are of the physical geography, of technical practices, of cultural customs, or of anything else. I had to rely on wonderfully helpful people to do that actual data collection; I was merely along for the ride, perhaps directing, but still little more than an observer. 