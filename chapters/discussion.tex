\chapter{Discussion}
\label{discussion}

\section{Examining the Kansas Signatures}
\label{discussion:kssigs}

The Kansas signatures extracted from the k-means clusters (\autoref{fig:KScropsigs}) were not as variable as I was expecting. Based on some initial testing results, presented in \autoref{appendix:testing:r3} beginning on \autopageref{appendix:testing:r3}, I expected to find some strange looking cluster signatures. Aside from the Soy\_1 signature, and perhaps the Corn\_1 signature to some degree, the cluster signatures were fairly typical in appearance. The Sorghum\_1 signature appears more or less as expected over the TSI's date range, but does seem to be missing the late-season downslope.

Using the k-means algorithm to cluster each crop's pixels might not have adequately captured the variability in the crop signatures as labeled by the CDL to allow the fit algorithm to match the CDL classification. That is, perhaps k-means separates pixels that would have similar RMSE values when fit with the same crop reference signature. Future research might want to consider clustering based on the RMSE value of each pixel to the others; pixels with low RMSE values when compared to one another would be grouped together, as they likely to be transformations of the same base temporal signature. Pixels with considerably different RMSE values would suggest different base temporal signatures.

Despite the fact the k-means might not capture similarity in the same way as the fit algorithm, it is worth noting that the k-means clusters do not divide any fields: each pixel in a field is assigned the same cluster (\autoref{map:KSclusters}). This result shows that each pixel in a field of multiple pixels typically has a similar signature to all the others in the field, at least using k-means as a measure of similarity. This result is further confirmation of the hypothesis that each pixel of a crop that grew under the same conditions, such as a field, should have the same development and therefore same temporal signature.

\section{Breakdown of the Kansas Classification}
\label{discussion:ksclassification}

The initial verification of the Kansas reference signatures, done by classifying the TSI of the Kansas study area, demonstrated the method performs well when used to match the original source data (\autoref{table:ksresults}). The 84.4 percent overall accuracy and kappa value of 0.76 are well within the range considered acceptable, especially given the CDL's published accuracy of 88.4 percent (to which this classification is compared).

Some confusion between corn and soy, as well as soy and ``other,'' pulled down the overall accuracy, as well as the producer and user accuracies of each of those classes. The similarities between the corn and soy signatures may cause late corn and early soy to be confused if the range of the offset of the beginning day of the reference signature, $tshift$, allows overlap between the two. I believe much of the soy and ``other'' confusion was due to the Soy\_1 reference signature (\autoref{fig:KScropsigs}). Examining the CDL classes of the best fit pixels in the RMSE raster showed a number of grassland pixels were well fit by that particular signature. However, omitting that particular signature seemed to allow one of the corn signatures to take many of the soy pixels, which I find strange due to the peculiar shape of the Soy\_1 signature. In fact, the signature does not match the traditional soy signature, and makes me question the validity of the CDL in this case (also see \autoref{appendix:testing:r3} for more CDL problems, and \autoref{appendix:cdl} for notes about the CDL).

Classifying sorghum did not seem to be very effective; only two of the 18 sorghum pixels were accurately identified. The sorghum RMSE raster had the lowest threshold value at 450, but increasing the threshold only caused greater class confusion. Omitting the signature entirely had a slight negative effect on the overall accuracy, as the ``other'' pixels it misclassifies would otherwise be misclassified by corn and soy. It is possible that I should have added another 16-day composite to the TSI to capture the tail-end of the sorghum signature, though I believe doing so would have caused more harm as winter crop plantings would begin to interfere with summer crop signatures (see the ``end-of-year bump'' discussion in \autoref{appendix:testing:r4}). Moreover, the similarities between sorghum and soy signatures might cause confusion if the sorghum threshold were to be raised. However, due to the low number of sorghum pixels, the validity of any conclusions about classifying this particular crop is questionable. A larger sample size and more testing are required.

\section{The Pellegrini Classification and Class Confusion}

As shown in \autoref{map:ARclassification}, classifying the Argentina TSI with the reference signatures from the k-means clustering of the Kansas TSI was able to effectively separate areas of the summer crops corn, soy, sorghum, and poroto from most other land cover classes, but classified those summer crop pixels predominately as as corn. While \autoref{table:ARbestresult} reflects this class confusion in the producer and user accuracies, the low sample count for corn, soy, and sorghum compared to all the other land covers deceptively skews the overall accuracy higher. \autoref{table:ARpurepxresults} is the confusion matrix for the same classification, but compared to the entire ground truth dataset, encompassing some 85,821 pixels, instead of the rather limited 378 random sample points. While the overall accuracy actually improves slightly with this new reference dataset, it must be noted that neither of these accuracy assessments is able to account for errors resulting from mixels. However, the increased number of samples better demonstrates the significant corn-soy confusion. For instance, of the 6,621 pixels classified as corn, 2,076 are soy. Errors of omission are also more prominent: 2,234 of the 5,706 corn pixels were left as ``other.'' A similar proportion of soy pixels were also classified ``other.'' Increasing the RMSE thresholds on the RMSE rasters to decrease these omissions only increased the errors of commission, confusing ``other'' pixels for crops. The low kappa statistic of both accuracy assessments, 0.54 and 0.51 respectively, is reflective of the poor accuracies within the summer crops.

\begin{sstable}
  \centering
  \caption{Summer 2014 Pellegrini Best Classification Accuracy Checked Against All Pure Pixels}
  \label{table:ARpurepxresults}
  \begin{tabu}{rrrrrrrl}
    \toprule
     & & \multicolumn{4}{c}{\textbf{Reference Data}} & & \\
     &  & Corn & Soy & Sorghum & Other & Total & User Acc. \\
    \midrule
    \multirow{4}{*}{\rotatebox{90}{\textbf{Classified}}} & Corn & 3283 & 2076 & 61 & 1201 & 6621 & 49.58\% \\
     & Soy & 189 & 313 & 36 & 458 & 996 & 31.43\% \\
     & Sorghum & 0 & 0 & 0 & 0 & 0 & 0.00\% \\
     & Other & 2234 & 1523 & 60 & 74387 & 78204 & 95.12\% \\
     & Total & 5706 & 3912 & 157 & 76046 & 85821 &  \\
     & Producer Acc. & 57.54\% & 8.00\% & 0.00\% & 97.82\% &  &  \\
    \multicolumn{8}{r}{Overall Accuracy: 90.87\%} \\
    \multicolumn{8}{r}{Kappa: 0.51} \\
    \bottomrule
  \end{tabu}
\end{sstable}

Considering that the ``other'' pixels contain a number of different classes, I found the frequency of corn and soy classifications within each of the ``other'' land cover classes. The results of this analysis (\autoref{table:ARotherconfusion}) show that the main sources of confusion were, from greatest to least, the true ``other land cover'' class, pasture, and poroto. However, when finding the percent of the land cover class pixels that were confused, poroto leads with over 26 percent of its pixels confused for either corn or soy. This confusion, and the confusion of some pasture areas as well, does make some sense, as these land covers, soy, and corn are all planted in spring and reach peak maturation during the summer months. Depending on the type of pasture, it may or may not grow back after cutting; if it does not, the temporal signature may bear some resemblance to corn or soy.\footnote{I heard a few names for a few different types of pasture grasses that were being grown in the area, but I believe the two most prevalent are known locally as \textit{\textspanish{gatom pani}} and \textit{\textspanish{grama}}. I have not been able to determine if \textit{\textspanish{gatom pani}} has an english name; it is possible that I did not get the correct spelling. Blue Grama Grass (textit{Bouteloua gracilis}) is a common forage grass native to North America, though I am unsure if it is the same plant grown in Pellegrini. I did not learn much about any other pasture grasses cultivated in the area, or about typical harvesting practices.}

\begin{table}[b]
  \begin{Spacing}{1.0}
  \centering
  \caption{Pellegrini Corn and Soy Confusion with ``Other'' Land Cover Classes}
  \label{table:ARotherconfusion}
  \begin{tabu}{X[0.6,m,c]X[0.5,m,c]|X[1,m,c]X[0.55,m,c]|X[1,m,c]X[0.55,m,c]}
    \toprule
    \textbf{Land Cover} & \textbf{Total Pixels} & \textbf{Confused as Corn} & \textbf{Percent of Total} & \textbf{Confused as Soy} & \textbf{Percent of Total} \\
    \midrule
    Forested & 63,978 & 194 & 0.30 & 26 & 0.04 \\
    Other & 5,393 & 306 & 5.67 & 322 & 5.97 \\
    Pasture & 5,252 & 396 & 7.54 & 50 & 0.95 \\
    Poroto & 1,369 & 303 & 22.13 & 59 & 4.31 \\
    Nothing & 485 & 2 & 0.41 & 1 & 0.21 \\
    \bottomrule
  \end{tabu}
  \end{Spacing}
\end{table}

From their appearance in Landsat imagery, the main locations where the other class was confused for corn and soy seem to be bare earth, possibly due to high soil salinity. The areas have low-to-moderate reflectivity in the visual bands, high reflectivity in the mid-infrared, and low reflectivity in the near-infrared, and do not exhibit much change over time. However, the plots of a random sampling of pixels from the TSI show temporal signatures like that in \autoref{fig:ARweirdsig}.
I am currently unable to explain what is in these areas or why they are confused for corn and soy. I believe there may be some sort of summer grass cover or other seasonal vegetation, but I cannot understand the lack of near-infrared reflectance as observed in Landsat images from multiple dates throughout the summer.

\begin{ssfigure}
  \centering
  \input{plots/strangepoint1.pgf}
  \caption{Signature of an Unknown Pixel Confused for Corn and Soy in Pellegrini}
  \label{fig:ARweirdsig}
\end{ssfigure}


\section{Clustering Pellegrini}

To further examine the class confusion in Pellegrini, I used the same k-means clustering as in Kansas to identify the three main signatures for each of the eight land cover classes in the ground truth dataset. The extracted corn, soy, and sorghum signatures are shown in \autoref{fig:ARcropsigs}, while the poroto and pasture signatures are in \autoref{fig:ARporotopasturesigs}.

The cause of the corn-soy confusion is immediately visible: both crops peak around the same date. I did not expect this result, as the typical planting dates I collected suggest soy should peak earlier than corn.\footnote{Even if soy had peaked earlier than corn, I would not have been able to achieve an accurate classification with the current tools, as corn-before-soy is a key assumption. That is, a single $tshift$ parameter is specified for all reference signatures. It would be possible to rewrite the tool to allow different $tshift$ values for different signatures, which might have sufficed if soy did peak before corn, but does not help when the peaks are coincident.} However, I also gathered that precipitation was the limiting factor in planting (confirmed by \autocite{sacks2010crop}), and often farmers will wait until a certain amount of rain has fallen before planting. In fact, I was told the rains this year were quite late, and on multiple occasions farmers told me they had planted a field late due to lack of rain. While I must admit I am not a farmer and do not know this for certain, the reason for waiting did not seem to be out of concern for plant health, in that too early of planting would negatively impact the crop's health, but seemed to be primarily economic: farmers do not want to pay to plant crops that will not grow if the rains never come.

\begin{ssfigure}
  \centering
  \input{plots/cropsigsAR.pgf}
  \caption{Corn, Soy, and Sorghum Signatures Extracted from the Pellegrini TSI}
  \medskip
  \small
  These are the signatures from the k-means crop clusters found in the Pellegrini TSI. While some strange exceptions like Soy\_3, Sorghum\_1, and Sorghum\_3 deviate from the rest, the overwhelming similarities between signatures of different crops is striking. Unlike the crop signatures from Kansas, Pellegrini's crops are not temporally separated, but peak almost simultaneously.
  \label{fig:ARcropsigs}
\end{ssfigure}

\begin{ssfigure}
  \centering
  \input{plots/poroto_pasture_sigsAR.pgf}
  \caption{Poroto and Pasture Signatures Extracted from the Pellegrini TSI}
  \label{fig:ARporotopasturesigs}
\end{ssfigure}
The forest, nothing, and other signatures are in \autoref{fig:ARothersigs}.

\begin{ssfigure}
  \centering
  \input{plots/othersigsAR.pgf}
  \caption{Forested, ``Nothing,'' and ``Other'' Signatures Extracted from the Pellegrini TSI}
  \label{fig:ARothersigs}
\end{ssfigure}

I also observed fields of corn, soy, sorghum, and poroto in many different states of development, from the early, barely-germinated stage to the late, fruit-bearing stage. To me, the broad range of development is likely because water is the only limiting factor defining the typical planting dates. That is, the threat of changing temperatures does not affect planting decisions like in Kansas, and farmers have much more flexibility (this flexibility has been observed in other regions per \textcite{sacks2010crop}). Additionally, early and late planted crops should mature about the same time, due to their development being limited by water. My method should be able to accommodate the range of planting and maturation dates by adjusting the bounds on the $tshift$ parameter, allowing the reference signatures more freedom to temporally align themselves with the pixel values. However, there is one problem with that idea: my method deals with similar-looking signatures, like corn and soy, by making an assumption that each should peak within different time ranges. In Kansas, corn peaks before soy, and as such they can be differentiated. Planting dates for Argentina as a whole also suggest corn should peak before soy. However, in Pellegrini specifically, the weather conditions and consequent agricultural practices do not allow this assumption to be met. When both crops peak about the same date, their temporal signatures are not significantly different, leading to the class confusion exhibited.

I believe higher temporal resolution data might create more detailed temporal signatures, which could allow for more difference to be detected between different crops. Combining such data with certain noise-filtering methods may allow signatures to be smoothed in ways that might accentuate differences between similar crops like corn and soy (see \textcite{doraiswamy2006improved} and \textcite{sakamoto2010a-two-step} for examples of higher temporal resolution data and filtering).