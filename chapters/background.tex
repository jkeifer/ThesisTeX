\chapter{Background}
\section{Deforestation and the \textit{Ley de Bosques} (Forest Act) in Argentina}

The conversion of forestland to other uses has seriously impacted Argentina’s forests. In 1915 it was remarked that 30 percent of the country had forest cover, but in 2001 only 10 percent remained forested \autocite{secretaria-de-d2001primer}. Over the period 1998 to 2002, Argentina lost over 940,000 hectares of forest cover \autocite{secretaria-de-a2007informe}. The high rate of deforestation concerned policymakers, and Law 26.331, or the \textit{Ley de Bosques} (Forest Act), was voted into law in 2007 in an effort to preserve remaining native forests. Areas of native forest are defined to be those with forest cover of at least 20 percent native species, and that have trees of a minimum of 7 meters high. The law designates red, yellow, and green areas, each with different restrictions on clearing and use. Red is assigned to areas of “high conservation value,” yellow is for areas that must be managed sustainably, and green allows “partial or total use” \autocite[25]{gulezian2009environmental}. Each provincial government was responsible for determining how to classify their native forest area, and each enacted the \textit{Ley de Bosques} regulations under the \textit{Ordenamiento Territorial de los Bosques Nativos} (Land Management Order for Native Forests, OTBN).

As a part of Law 26.331, ongoing land cover studies are done to examine the effectiveness of the legislation. Between 2006 and the passing of the law, 573,296 hectares of native forest cover were lost. From the passing of the law in 2007 and the classification of the OTBN areas in 2009, a further 473,001 hectares were deforested. From the enacting of the OTBN (in 2009) and 2011, some 459,108 hectares were found to have been lost \autocite{secreteria-de-a2012monitoreo}. The continued deforestation suggests that, in the context of the native forest areas, the \textit{Ley de Bosques} may have had a small effect in reducing deforestation, but overall levels still remain quite high. Consequently, some have begun to question the effectiveness of the law at slowing cutting \autocites{valpreda2012the-protection}{greenpeace-arge2013ley-de-bosques:}. Clearly, a better understanding of the driving forces of deforestation in Argentina needs to be developed.

\section{Soy and its effects}

The increase of soybean in Argentina has occurred at a rapid pace throughout the last two decades, making it the third largest producer of soy in the world \autocite{us-foreign-agri2013world}. Necessarily, as soy production rises, so does its spatial extent and the intensity of cultivation methods. Currently, almost all of Argentina’s soy production is using genetically modified (GM) varieties, specifically Monsanto’s “Roundup Ready” beans \autocite{greenpeace-inte2005the-expanding}. The highly mechanized and input intensive nature of this crop type calls into question other environmental consequences of soybean cultivation, such as pesticide runoff, glyphosate-resistant weeds, and soil depletion \autocite{pengue2005transgenic}.

A number of studies have addressed soy and deforestation in Northwest Argentina, but only one has used methods capable of mapping crop types in deforested areas \autocite{volante2005analisis}. However, this study by the Argentine \textit{Instituto Nacional de Tecnología Agropecuraia} (National Institute of Agricultural Technology, INTA) does not have well-documented methodology and has not been updated since 2005. Of the remainder, all used remote sensing techniques to classify only LULC and not specific crop types, leaving the effect of soy on LULC as an underlying assumption \autocites{grau2005agriculture}
{grau2008balancing}{grau2005globalization}
{boletta2006assessing}{gasparri2009deforestation}. While the extreme deforestation in Argentina is undeniable---and certainly soy plays a part---its role has not been examined in full, leaving unsubstantiated the perception of soy as the driving force in this process.

The goal of this research is to develop an image classification capable of mapping agricultural crops by type, allowing soy to be explicitly identified on remotely sensed imagery. The accurate and efficient mapping of soy distributions and their changes over time could allow further investigation of the roles of soy in deforestation. The direct and indirect effects soy crops have had on deforestation can thus be understood conceptually and systemically at both regional and local scales, which could lead to the development of more effective policies for land management \autocite{brown2007multitemporal}.
