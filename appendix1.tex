\newpage
\section*{Appendix 1 -- Project Code}

The code for this project is all written in python version 2.7.5, and has various dependencies, listed below:

\begin{spacing}{1}
\begin{itemize}
  \item[--] GDAL 1.8 with python bindings
  \item[--] Scipy 0.12.0
  \item[--] Numpy 1.6.2
\end{itemize}
\end{spacing}

The project is currently split into three scripts: one to import the VI images and build the composite multi-date VI image (Script 1); one to extract the mean reference curves from pixels of known crop type in an image (Script 2, page \pageref{script2}); and one to use minimization to fit said reference curves to pixel curves, writing the RMSE to output images and finding the best fit (Script 3, page \pageref{script3}).


\subsubsection*{Script 1: Building multi-date images}\label{script1}
\begin{minted}[baselinestretch=1.2, linenos, numbersep=5pt, samepage=false, fontsize=\footnotesize]{python}
from osgeo import gdal
from osgeo.gdalconst import *
import os, sys

rootDIR = "/Users/phoetrymaster/Documents/School/Geography/Thesis/Data/MODIS_KANSAS_2012/"
#rootDIR = "/Users/phoetrymaster/Documents/School/Geography/Thesis/Data/MODIS 7_2012-2013/"
outName = "test"

newfoldername = "kansas"

find = "EVI"
ext = ".hdf"
drivercode = "ENVI"
ndvalue = -3000
projection = "PROJCS[\"Sinusoidal\",GEOGCS[\"GCS_Undefined\",DATUM[\"D_Undefined\",SPHEROID[\"User_Defined_Spheroid\",6371007.181,0.0]],PRIMEM[\"Greenwich\",0.0],UNIT[\"Degree\",0.017453292519943295]],PROJECTION[\"Sinusoidal\"],PARAMETER[\"False_Easting\",0.0],PARAMETER[\"False_Northing\",0.0],PARAMETER[\"Central_Meridian\",0.0],UNIT[\"Meter\",1.0]]"


########## METHODS ##########


def find_files(searchdir, ext):
    hdfs = []

    for root, dirs, files in os.walk(searchdir):
        for f in files:
            if f.upper().endswith(ext.upper()):
                foundfile = os.path.join(root, f)
                hdfs.append(foundfile)

    return hdfs


def create_output_dir(root, name):
    dirpath = os.path.join(root, name)

    if os.path.isdir(dirpath):
        count = 1
        dirpath_ = dirpath + "_"
        while 1:
            dirpath = dirpath_ + str(count)
            count += 1
            if not os.path.isdir(dirpath):
                break

    os.makedirs(dirpath)
    return dirpath


def create_output_raster(outFile, cols, rows, bands, datatype, drivername="GTiff"):
    driver = gdal.GetDriverByName(drivername)
    driver.Register()

    outds = driver.Create(outFile, cols, rows, bands, datatype)

    return outds


def get_output_params(filepath):
    image = gdal.Open(filepath, GA_ReadOnly)

    if image is None:
        raise Exception("Could not open " + filepath)

    rows = image.RasterYSize
    cols = image.RasterXSize
    band = image.GetRasterBand(1)
    bandtype = band.DataType
    geotransform = image.GetGeoTransform()
    projection2 = image.GetProjection()

    image = ""

    return rows, cols, bandtype, geotransform, projection


def get_hdf_subdatasets(hdfpath):
    hdf = gdal.Open(hdfpath, GA_ReadOnly)

    if hdf is None:
        raise Exception("Could not open " + hdfpath)

    sds = []
    hdfsds = hdf.GetSubDatasets()

    for data in hdfsds:
        sds.append((data[0], data[0].split(" ")[-1]))

    hdf = ""

    return sds


def main():
    outdir = create_output_dir(rootDIR, newfoldername)
    print "\nOutputting files to : {0}".format(outdir)

    print "\nFinding HDF files in directory/subfolders: {0}".format(rootDIR)
    hdfs = find_files(rootDIR, ext)
    print "\tFound {0} files.".format(len(hdfs))

    print "\nGetting images to process of type {0}...".format(find)
    toprocess = []

    for hdf in hdfs:
        sds = get_hdf_subdatasets(hdf)
        for ds in sds:
            if find.upper() in ds[1].upper():
                toprocess.append(ds[0])
                print "\t\t{0}".format(ds[0])

    bands = len(toprocess)
    print "\tFound {0} images of type {1}.".format(bands, find)

    print "\nGetting output parameters..."
    rows, cols, datatype, geotransform, projection = get_output_params(toprocess[0])
    print "\tParameters: rows: {0}, cols: {1}, datatype: {2}, projection: {3}.".format(rows, cols, datatype, projection)

    outfile = os.path.join(outdir, outName) + ".tif"
    print "\nOutput file is: {0}".format(outfile)

    outds = create_output_raster(outfile, cols, rows, bands, datatype, drivername=drivercode)
    print "\tCreated output file."

    print"\nAdding bands to output file..."
    for i in range(0, bands):
        print "\tProcessing band {0} of {1}...".format(i + 1, bands)
        image = gdal.Open(toprocess[i])
        band = image.GetRasterBand(1)

        outband = outds.GetRasterBand(i + 1)

        print "\t\tReading band data to array..."
        data = band.ReadAsArray(0, 0, cols, rows)

        print "\t\tWriting band data to output band..."
        outband.WriteArray(data, 0, 0)
        outband.SetNoDataValue(ndvalue)
        outband.FlushCache()

        del data, outband
        image = ""

    print "\tFinished adding bands to output file."

    print "\nSetting transform and projection..."
    outds.SetGeoTransform(geotransform)
    outds.SetProjection(projection)

    outDS = ""

    print "\nProcess completed."


########## PROCEDURE ##########


if __name__ == '__main__':
    sys.exit(main())
\end{minted}

\subsubsection*{Script 2: Getting reference curves from known pixels}\label{script2}
\begin{minted}[baselinestretch=1.2, linenos, numbersep=5pt, samepage=false, fontsize=\footnotesize]{python}
from osgeo import gdal
from osgeo.gdalconst import *
from math import floor

imagepath = "/Volumes/J_KEIFER/Thesis/Data/ARC_Testing/test1.dat"

soylocs = [(6002, 2143), (5944, 2102), (5746, 2183), (5998, 2171)]
cornlocs = [(5997, 2139), (5940, 2096), (6051, 2230), (5691, 1998)]
wheatlocs = [(5993, 2136), (5937, 2080), (5935, 2076), (5921, 2217)]
refstoget = {"soy": soylocs, "corn": cornlocs, "wheat": wheatlocs}


gdal.AllRegister()

img = gdal.Open(imagepath, GA_ReadOnly)

if img is None:
    raise Exception("Could not open " + imagepath)

bands = img.RasterCount

print "Found {0} bands in input image.".format(bands)

refs = {}

for key, val in refstoget.items():
    print "Processing {0} coordinates:".format(key)
    dict = {}
    for i in range(0, bands):
        band = img.GetRasterBand(i+1)
        print "\tProcessing band {0}".format(i+1)
        values = []
        for loc in val:
            print "\t\tGetting position {0}".format(loc)
            values.append(int(band.ReadAsArray(int(floor(loc[0])), int(floor(loc[1])), 1, 1)))
        dict[(i*16+1)] = sum(values) / float(len(values))
        band = ""
    refs[key] = dict

print refs

img = ""
\end{minted}

\subsubsection*{Script 3: Using reference curves to calculate fit for each pixel}\label{script3}
\begin{minted}[baselinestretch=1.2, linenos, numbersep=5pt, samepage=false, fontsize=\footnotesize]{python}
from osgeo import gdal
from osgeo.gdalconst import *
import os
import numpy
from numpy import sum
from scipy import interpolate
from scipy import optimize

gdal.UseExceptions()

imagepath = "/Users/phoetrymaster/Documents/School/Geography/Thesis/Data/ARC_Testing/ClipTesting/ENVI_1/test_clip_envi_3.dat"
rootdir = "/Users/phoetrymaster/Documents/School/Geography/Thesis/Data/OutImages/"

newfoldername = "Testing"

drivercode = 'ENVI'
ndvalue = -3000

startDOY = 1
thresh = 500
bestguess = 0
fitmthd = 'SLSQP'


refs = {
    'soy': {1: 174.5, 97: 1252.25, 65: 1139.5, 209: 7659.0, 273: 4606.75, 337: 1371.75, 17: 1055.5, 33: 1098.0,
            49: 1355.25,
            129: 1784.75, 257: 6418.0, 321: 1644.5, 305: 1472.75, 193: 5119.75, 289: 1878.75, 177: 3439.5, 241: 7565.75,
            81: 1205.5, 225: 7729.75, 145: 1736.25, 161: 1708.25, 353: 1358.25, 113: 1340.0},
    'corn': {1: 392.25, 97: 1433.25, 65: 1258.5, 209: 6530.0, 273: 1982.5, 337: 1658.5, 17: 1179.25, 33: 1196.75,
             49: 1441.25, 129: 1885.25, 257: 2490.25, 321: 1665.75, 305: 1439.0, 193: 6728.25, 289: 1634.5,
             177: 6356.75,
             241: 4827.25, 81: 1355.75, 225: 5547.5, 145: 2196.5, 161: 3143.25, 353: 1704.75, 113: 1716.5},
    'wheat': {1: 719.75, 97: 6594.75, 65: 1935.25, 209: 2013.5, 273: 1493.5, 337: 1498.25, 17: 1816.5, 33: 1815.0,
              49: 1985.25, 129: 6758.0, 257: 1685.75, 321: 1582.5, 305: 1163.25, 193: 2186.25, 289: 1264.5, 177: 2222.5,
              241: 2301.0, 81: 4070.5, 225: 1858.0, 145: 6228.5, 161: 3296.5, 353: 1372.5, 113: 7035.25}
}


########## METHODS ##########


def create_output_raster(outFile, cols, rows, bands, datatype, drivername="GTiff"):
    driver = gdal.GetDriverByName(drivername)
    driver.Register()

    outds = driver.Create(outFile, cols, rows, bands, datatype)

    return outds


def create_output_dir(root, name):
    dirpath = os.path.join(root, name)

    if os.path.isdir(dirpath):
        count = 1
        dirpath_ = dirpath + "_"
        while 1:
            dirpath = dirpath_ + str(count)
            count += 1
            if not os.path.isdir(dirpath):
                break

    os.makedirs(dirpath)
    return dirpath


def get_sort_dates_values(vals, threshhold=-3000):
    """Gets the DOY dates (the keys) in a list from dictonary vals and sorts those, placing them in chronological order
    (list x0). Then the function iterates over these values and gets the corresponding values, thresholding values if
    they are lower than an optional threshhold value (-3000 default = NoData in MODIS imagery), then appending them to
    the list y. x and y are then returned."""

    x = vals.keys()
    x.sort()
    y = []

    for i in x:
        if vals[i] < threshhold:
            y.append(threshhold)
        else:
            y.append(vals[i])

    return x, y

def find_fit(valsf, valsh, bestguess, threshhold, mthd="TNC"):

    x0, y0 = get_sort_dates_values(valsf, threshhold=threshhold)
    x1, y1 = get_sort_dates_values(valsh)

    tck = interpolate.splrep(x1, y1)

    fun = lambda x: ((1 / 22.8125 * sum(
        (valsf[i] - (x[0] * interpolate.splev((x[1] * (i + x[2])), tck))) ** 2 for i in x0)) ** (
                         1. / 2))

    bnds = ((0.6, 1.4), (0.6, 1.4), (-10, 10))

    res = optimize.minimize(fun, (1, 1, bestguess), method=mthd, bounds=bnds)

    return res.fun, res.x, res.message


########## PROCEDURE ##########


try:
    outdir = create_output_dir(rootdir, newfoldername)
    print "\nOutputting files to : {0}".format(outdir)

    gdal.AllRegister()

    #Open multi-date image to analyze
    img = gdal.Open(imagepath, GA_ReadOnly)

    if img is None:
        raise Exception("Could not open: {0}".format(imagepath))

    #Get image properties
    cols = img.RasterYSize
    rows = img.RasterXSize
    bands = img.RasterCount
    geotransform = img.GetGeoTransform()
    projection = img.GetProjection()

    print "Input image dimensions are {0} columns by {1} rows and contains {2} bands.".format(cols, rows, bands)


    #Create output rasters for each crop type to hold residual values from fit and arrays
    print "\nCreating output files..."
    outfiles = {}
    outdatasets = {}
    outarrays = {}
    for key in refs:
        outfile = os.path.join(outdir, key) + ".tif"
        outfiles[key] = create_output_raster(outfile, cols, rows, 1, GDT_Float32, drivername=drivercode)
        outfiles[key].SetGeoTransform(geotransform)
        outfiles[key].SetProjection(projection)
        outdatasets[key] = outfiles[key].GetRasterBand(1)
        outarrays[key] = numpy.zeros(shape=(rows, cols))
        print "\tCreated file: {0}".format(outfile)

    #Create output raster for bestFit
    outfile = os.path.join(outdir, "bestFit") + ".tif"
    fitimgfile = create_output_raster(outfile, cols, rows, 1, GDT_Byte, drivername=drivercode)
    fitimgfile.SetGeoTransform(geotransform)
    fitimgfile.SetProjection(projection)
    fitimg = fitimgfile.GetRasterBand(1)
    fitarray = numpy.zeros(shape=(rows, cols))
    print "\tCreated file: {0}".format(outfile)


    #Iterate through each pixel and calculate the fit for each ref curve; write residuals and best fit to rasters
    for row in range(0, rows):
        for col in range(0, cols):
            valsf = {}
            print "Pixel r:{0}, c:{1}:".format(row, col)
            for i in range(0, bands):
                band = img.GetRasterBand(i+1)
                measured = int(band.ReadAsArray(col, row, 1, 1))
                valsf[startDOY + i*16] = measured
            count = 1
            fit = {}
            for key, val in refs.items():
                res, transforms, message = find_fit(valsf, val, bestguess, threshhold=thresh, mthd=fitmthd)
                outarrays[key][row, col] = res
                print "\t{0}: {1}, {2}, {3}".format(key, res, transforms, message)
                fit[res] = count
                count += 1
            fitarray[row, col] = fit[min(fit.keys())]

    #Write output array values to files
    print "Writing output files..."
    for key, values in outdatasets.items():
        outdatasets[key].WriteArray(outarrays[key], 0, 0)

    fitimg.WriteArray(fitarray, 0, 0)
    print "\nProcess finished."

except Exception as e:
    print e

finally:
    print "\nClosing files..."
    try:
        fitimg = None
        fitimgfile = None
    except:
        pass
    try:
        for key, value in outdatasets.items():
            outdatasets[key] = None
    except:
        pass
    try:
        for key, value in outfiles.items():
            outfiles[key] = None
    except:
        pass
\end{minted}
